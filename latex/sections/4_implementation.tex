\chapter{Implementing Drivel}\label{ch:implementation}

This chapter describes implementation efforts, challenges, and noteworthy information for implementers, including deployment steps and practical considerations such as traffic shaping.
\Cref{sec:tech-stack} gives an overview of the technology stack, languages, and libraries used in the implementation, with a particular focus on the changes made to the preexisting lyrebird project \cite{lyrebird} to support \drivel{}. This section also describes the environment in which the protocol was developed and tested.
\Cref{sec:parameters} outlines the process for choosing concrete cryptographic parameters, focusing on the trade-offs between security and traffic volume.
\Cref{sec:drivel-instance} quickly discusses how we chose to instantiate the generic primitives used in \drivel{}.
\Cref{sec:impl-architecture} describes how the protocol design was mapped to the actual software components, including noteworthy architectural changes and cryptographic interfaces.
\Cref{sec:deployment} provides practical guidance for the deployment and use of the system, with attention to real-world integration barriers.
Finally, \cref{sec:challenges-learnings} discusses the key challenges encountered during development and how they were addressed.

Later on, \cref{ch:protocol-changes} will present, among other proposed changes, a detailed explanation of how handshake messages could be fragmented to enhance traffic shaping capabilities.

\section{Technology Stack and Environment} \label{sec:tech-stack}

The implementation created as part of this thesis is available via GitLab. It consists of two parts: A fork of the lyrebird project \cite{lyrebird} where the \drivel{} pluggable transport is implemented\footnote{See \url{https://gitlab.ethz.ch/himarc/lyrebird-pq-obfs}}, and a larger repository which includes a testing and benchmarking framework, analysis code and TeX sources of this document\footnote{See \url{https://gitlab.ethz.ch/himarc/thesis-quantumsafe-fep}}.

\paragraph{Structure of the lyrebird Fork and Dependencies}

The \drivel{} implementation, like the other six pluggable transports in the lyrebird repository, is written in Go. The lyrebird repository contains several Go modules:
\begin{itemize}
    \item \texttt{cmd} implements general pluggable transport logic and represents the program entry point.
    
    \item \texttt{common} contains various useful implementations such as cryptographically secure random number generation (CSRNG), deterministic random bit generation (DRBG), logging, probability distributions, filter objects to prevent replay attacks, SOCKS5 implementation, etc. These implementations may be reused outside of the lyrebird repository.
    
    \item \texttt{internal} contains implementations that are only reusable within lyrebird. The upstream repository only contained an X25519 and \textsf{Elligator2}~\cite{CCS:BHKL13} implementation in this module.
    
    \item \texttt{transports} finally has submodules for each pluggable transport that each implement a common interface used by the \texttt{cmd} module.
\end{itemize}

During the course of this thesis, the \texttt{internal} module was significantly refactored and extended. Furthermore, a new submodule \texttt{transports/drivel} was added to the \texttt{transports} module, based on a copy of the \texttt{transports/obfs4} module. We refer readers to \cref{sec:impl-architecture} for details about these changes.

External Go dependencies required for \drivel{} and not specific to other pluggable transports are limited to the following modules and versions:
\begin{itemize}
    \item \gopkgref{filippo.io/edwards25519}{v1.1.0} for easier handling of elliptic-curve cryptography,
    
    \item \gopkgref{gitlab.com/yawning/edwards25519-extra}{v0.0.0-20231005122941-2149dcafc266} for its implementation of \textsf{Elligator2}~\cite{CCS:BHKL13},
    
    \item \gopkgref{github.com/dchest/siphash}{v1.2.3} for the pseudorandom function employed in the DRBG,
    
    \item \gopkgref{github.com/open-quantum-safe/liboqs-go}{v0.0.0-20250119172907-28b5301df438} as a Go wrapper around the C library \texttt{liboqs} which provides many KEM implementations,
    
    \item \gopkgref{gitlab.torproject.org/tpo/anti-censorship/pluggable-transports/goptlib}{v1.6.0} for handling arguments passed to the pluggable transport, and
    
    \item \gopkgref{golang.org/x/crypto}{v0.32.0} for its implementation of some elliptic-curve operations, HKDF and NaCl secretbox.
\end{itemize}

Other than \texttt{liboqs-go}, all of these dependencies were present in the upstream repository and required for \obfsfour{}.

In addition to the Go modules above, \drivel{} also depends on the \href{https://github.com/open-quantum-safe/liboqs}{liboqs} C library, compiled statically from source during the build process. This archive is then included in a local build of \texttt{liboqs-go}, removing the need for lyrebird users to install \texttt{liboqs} separately.

However, the inclusion of C code also means that a fully static compilation of the forked lyrebird repository was not possible anymore in Go's default build process. The upstream repository accomplished this via the use of the flag \texttt{CGO\_ENABLED}. Instead, the final executable now requires the following dynamically linked libraries to be present on the system:
\begin{itemize}
    \item \texttt{linux-vdso.so.1}
    \item \texttt{libresolv.so.2}
    \item \texttt{libpthread.so.0}
    \item \texttt{libc.so.6}
    \item \texttt{/lib64/ld-linux-x86-64.so.2}
\end{itemize}

To run end-to-end tests, the official bridge setup guide \cite{tor-bridge-setup} refers, among other options, to a Docker container \cite{tor-bridge-docker}.
We were able to adapt the container's configuration to use our forked repository and our modified build process (which now includes building \texttt{liboqs}) instead of the upstream version. The required dynamic libraries were already present on the system, demonstrating that deployment should not be hindered by our modifications to the dependencies and executable.

\paragraph{Testing and Benchmarking}

To perform unit and integration tests of the different software components, Go's built-in support for testing and benchmarking files was used. We have transferred all existing test cases for \obfsfour{} and extended them to include all new software components, and cover all enabled KEM/OKEM combinations.

In particular, cryptographic operations within \drivel{} are tested using positive and negative test cases on different levels. Unit tests verify the submodule's correctness, and integration tests simulate entire handshakes, including (de-)serialization of messages to and from byte strings.

The tests include a small variant that performs fewer tests of randomized functions, making a trade-off between execution time and completeness. All tests pass for the entire forked repository, also in their more extensive variant.

Benchmarks are also performed on multiple levels, including single KEM and OKEM operations (key generation, encapsulation, and decapsulation), as well as the execution of a complete simulated handshake. The latter benchmark allows a direct comparison to the corresponding \obfsfour{} benchmark. Furthermore, profiling can easily be performed to identify bottlenecks in the benchmark's performance and to quantify contributions to its memory consumption.

The results were collected and analyzed in Python and R.
Further discussion of the results follows in \cref{ch:experiment}.

\section{Choosing Parameter Sets} \label{sec:parameters}

\paragraph{Supported KEMs and OKEMs}
We begin with an overview of all supported KEMs and OKEMs in our implementation, including their parameter sets and corresponding NIST security levels 1, 3, and 5.

The following OKEMs are available for each NIST security level:
\begin{itemize}
    \item ML-Kemeleon, as in \cite{irtf-cfrg-kemeleon-00}, based on ML-KEM-512 (level 1), ML-KEM-768 (level 3), and ML-KEM-1024 (level 5)
    \item Unmodified FrodoKEM \cite{NISTPQC-R3:FrodoKEM20}, concretely FrodoKEM-640-AES (level 1), FrodoKEM-976-AES (level 3), and FrodoKEM-1344-AES (level 5)
    \item Unmodified Classic McEliece, as in \cref{sec:obfuscating-classic-mceliece}, concretely Classic-McEliece-348864 (level 1), Classic-McEliece-460896 (level 3), as well as Classic-McEliece-6688128, Classic-McEliece-6960119, and Classic-McEliece-8192128 (all three claiming level 5)
    \item $\feo[\mathsf{HQC}, E_\mathsf{HQC}]$, as in \cref{sec:obfuscating-hqc}, concretely based on HQC-128 (level 1), HQC-192 (level 3), and HQC-256 (level 5)
\end{itemize}

In addition, it is possible to use the implementation of X25519 and \textsf{Elligator2}~\cite{CCS:BHKL13} as an OKEM, but as this OKEM would not be resistant against quantum-computers, we only include it for the purposes of performance comparisons with \obfsfour{}.

For the selection of a KEM, the implementation supports the following for each NIST security level:
\begin{itemize}
    \item BIKE \cite{NISTPQC-R4:BIKE22}, concretely BIKE-L1 (level 1), BIKE-L3 (level 3), and BIKE-L5 (level 5)
    \item Streamlined NTRU Prime \cite{NISTPQC-R3:NTRUPrime20}, concretely sntrup761 (level 2)
    \item Unobfuscated variants for each OKEM listed above, except Classic-McEliece
\end{itemize}

Again, it is also possible to use a KEM based on X25519, which is not quantum-safe.

\paragraph{Disabled KEMs and OKEMs}

Due to the fact that our implementation relies on AES to instantiate the symmetric encryption component of \drivel{}, and to reduce the amount of options to consider in the following sections, we support only AES variants of FrodoKEM, even though SHAKE versions are offered by \texttt{liboqs}.
Similarly, but motivated also by a desire for a smaller traffic volume during the key exchange and a desire for smaller buffer sizes, Classic McEliece has been disabled as a KEM (though it remains available as an OKEM).
FrodoKEM variants using SHAKE, as well as unobfuscated Classic McEliece KEMs, could easily be re-enabled if so desired.

\paragraph{Selected Combinations}

Although \drivel{} can be instantiated with any choice of KEM or OKEM listed above, most combinations lead to inconsistent security levels or excessive message sizes. To reduce the number of combinations to a sensible amount, \cref{tab:drivel-params} defines a total of 12 sets of KEM/OKEM combinations, four for each of the NIST security levels 1, 3, and 5.

For the selection of parameter sets, we can observe that Classic-McEliece-6960119, and Classic-McEliece-8192128 will not selected as an OKEM because they achieve no higher NIST security level than Classic-McEliece-6688128 but use longer ciphertexts.
Further, we did not use sntrup761 as a KEM even though it would offer a 30\% reduction in size compared to BIKE-L1, as no other NTRU Prime parameter sets are available for NIST levels 3 and 5.

These parameter sets prioritize the selection of combinations that lead to the smallest required total handshake traffic, excluding padding.
\begin{table}
    \centering  \footnotesize
    \begin{tabular}{@{} L{0.1\textwidth-\tabcolsep} | L{0.16\textwidth-2\tabcolsep} L{0.17\textwidth-2\tabcolsep} L{0.32\textwidth-2\tabcolsep} R{0.11\textwidth-2\tabcolsep} R{0.14\textwidth-\tabcolsep} @{}}
    NIST Security\newline Level & Parameter Set\newline Name & KEM\newline Component & OKEM\newline Component & Total Traffic & First Fragment \\ \hline
    \multirow{4}{*}{Level 1} & drivel-L1a & ML-KEM-512 & ML-Kemeleon-512 & 2 541 & 893 \\
     & drivel-L1b & BIKE-L1 & ML-Kemeleon-512 & 4 087 & 893 \\
     & drivel-L1c & ML-KEM-512 & FEO-HQC-128 & 6 097 & 4 449 \\
     & drivel-L1d & ML-KEM-512 & Classic-McEliece-348864 & 1760 & 112 \\ \hline
    \multirow{4}{*}{Level 3} & drivel-L3a & ML-KEM-768 & ML-Kemeleon-768 & 3 620 & 1 268 \\
     & drivel-L3b & BIKE-L3 & ML-Kemeleon-768 & 7 546 & 1 268 \\
     & drivel-L3c & ML-KEM-768 & FEO-HQC-192 & 11 346 & 8 994 \\
     & drivel-L3d & ML-KEM-768 & Classic-McEliece-460896 & 2 524 & 172 \\ \hline
    \multirow{4}{*}{Level 5} & drivel-L5a & ML-KEM-1024 & ML-Kemeleon-1024 & 4 890 & 1 674 \\
     & drivel-L5b & BIKE-L5 & ML-Kemeleon-1024 & 12 030 & 1 674 \\
     & drivel-L5c & ML-KEM-1024 & FEO-HQC-256 & 17 653 & 14 437 \\
     & drivel-L5d & ML-KEM-1024 & Classic-McEliece-6688128 & 3 440 & 224
    \end{tabular}
    \caption[
        Definitions of parameter sets for \drivel{}, used in later experiments.
    ]{
        Definitions of parameter sets for \drivel{}, used in later experiments.
        Total traffic shows the combined number of bytes required across all handshake messages, excluding padding. First fragment denotes the size in bytes of just OKEM ciphertext and a 16 byte PRF value and thus illustrates the minimum amount of valid data required before the bridge may reply.
    }
    \label{tab:drivel-params}
\end{table}

Notably, HQC is not included as an unobfuscated KEM in any parameter set, due to its larger public key size compared to both ML-KEM and BIKE. Due to its more conservative assumptions, but consequently larger public key and ciphertext sizes, FrodoKEM is not included as KEM or OKEM in any parameter set. Depending on the assumptions one is comfortable with, ``a'' parameter sets may be suitable as only one KEM has to remain unbroken, whereas ``b'' can profit from different underlying assumptions for its KEM and OKEM, while ``c'' parameter sets feature a simpler encoding routine. Finally, ``d'' parameter sets employ Classic McEliece as an OKEM directly and require confidence in its pseudorandomness (see \cref{sec:obfuscating-classic-mceliece}), but offer significantly smaller message sizes.

When choosing between ``a'' and ``b'' parameter sets, it is worth considering the likelihood of the following scenario where a censor manages to break just one assumption underlying a used KEM or OKEM.
In an ``a'' parameter set, only the assumptions underlying ML-KEM are relevant, but a break could lead to a complete loss of confidentiality as censors might be able to infer the shared secrets generated in recorded key exchanges, decrypt all corresponding traffic and impersonate the bridge server from then on out.
In an ``b'' parameter set, the use of different underlying assumptions makes a break in any assumption more likely, but breaking only one assumption does not immediately mean a loss of all security guarantees: As an example, a censor that completely breaks an OKEM may impersonate bridge servers and retroactively identify obfuscated key exchanges, but cannot recover the KEM shared secrets and is thus not able to retroactively decrypt recorded traffic.

\section{Instantiating the Drivel Protocol} \label{sec:drivel-instance}

The specification of \drivel{} in \cref{fig:drivel} defines a handful of generic cryptographic primitives that need to be instantiated with concrete choices before implementation.
The authors offer some recommendations in \cite[Section~4.2]{EPRINT:GRSV25}.

We instantiate the primitives as follows:
\begin{itemize}
    \item the PRF $F_1$ is instantiated as \textsf{HDKF-Expand} with \textsf{SHA-256} \cite{C:Krawczyk10,rfc5869},
    
    \item the dual-PRF is instantiated as $F_2$ as \textsf{HMAC-SHA-256} \cite{C:BelCanKra96,KraBelCan97}, and
    
    \item the symmetric encryption scheme $SE$ is chosen to be \textsf{AES-256-CTR} with the appropriate key lengths $\keylen_1 = \keylen_2 = 256$ bits and a fixed IV of zero bytes.
\end{itemize}

We differ from the author's recommendations and the protocol specification only in two points:
\begin{enumerate}
    \item We do not use XOR as our one-time secure symmetric encryption scheme, and instead use AES in CTR mode. Using XOR would involve setting $\keylen_1 = \pklen$ and $\keylen_2 = \ctxtlen$, and the outputs of \textsf{HDKF-Expand} (used as $F_1$) are inherently limited to at most 8 160 bytes. This limitation would not affect any of our parameter sets proposed in \cref{sec:parameters}, but would lead to an avoidable loss of flexibility in the choice of KEM.

    \item The literal strings used for domain separation (such as e.g.~``:enckey1'') are prefixed with the string constant ``Drivel'', just as in \obfsfour{}. This cautionary measure helps to guarantee domain separation even in the case of reused keypairs over multiple protocol versions.
\end{enumerate}

\section{Core Architecture} \label{sec:impl-architecture}

Using these parameter sets and primitives to instantiate the \drivel{} protocol, we now take a closer look at how we implemented the \drivel{} protocol based on the existing \obfsfour{} implementation in the lyrebird repository.

Recall the general structure of the repository into four main Go modules from \cref{sec:tech-stack}. Here, we consider two of these modules in more detail: the \texttt{internal} module, which was significantly refactored and extended, and the \texttt{transports/drivel} module containing all \drivel{}-specific code.

\paragraph{Building a KEM and OKEM from the \texttt{internal/x25519ell2} Module}

In the implementation of \obfsfour{}, only one method of key exchange is available: Customized Diffie-Hellman key exchange on the elliptic curve Curve25519, possibly using the \textsf{Elligator2} encoding on the curve points transmitted over the wire.

Post-quantum key-exchange algorithms are almost universally specified using the syntax of KEMs. In order to compare \drivel{} performance in the traditional setting with its performance using post-quantum KEMs, we have extended the \texttt{internal/x25519ell2} module to implement a KEM and a filter-encode obfuscator performing \textsf{Elligator2}. \Cref{fig:impl-x25519-kem} shows pseudocode illustrating the usage of the module by \obfsfour{} (through its use of the \texttt{common/ntor} module), alongside the construction of our X25519 KEM.

As our goal is simply to show how the KEM construction was extracted from the existing code, we omit the full definition of \obfsfour{} here. Suffice to say, the protocol involves one static keypair $\pk_\mathsf{id}, \sk_\mathsf{id}$ (the public key is part of the bridge information) and two ephemeral keypairs generated during the handshake procedure. The static keypair has no encoding applied to the public key (as it is transmitted out-of-band), while the ephemeral keypairs also have obfuscated public keys. When all required values are available, methods $\mathsf{ServerHandshake}, \mathsf{ClientHandshake}$ are executed to compute shared secrets which in turn are used to authenticate and finalize the handshake.

Readers should note that, although the construction is similar in spirit to the DHKEM construction \cite[Section~4.1]{rfc9180}, this KEM has minor differences in the derivation of its shared secret. Instead, the focus was on staying as close as possible to the construction used in \obfsfour{}, in order to offer a fair comparison between the two pluggable transport constructions when targeting traditional security.

Unlike in the \obfsfour{} protocol, the newly constructed KEM has what initially appear to be slightly different semantics to generate the obfuscated ciphertext $\hat \pk$:
In \obfsfour{} and its method $\mathsf{NewKeypair}$, the value $u$ determined by $\sk$ is passed directly to $\mathsf{uToRepresentative}$ whereas in the KEM, additional serialization ($\mathsf{Bytes}$) and deserialization ($\mathsf{SetBytes}$) takes place before $\mathsf{uToRepresentative}$ is invoked.
We have experimentally verified that this does not change $u$ and that the same $\hat \pk$ values are generated with both approaches, given the same randomness $t$.
Finally, the randomness $t$ is not taken from the hash function output used to set $\sk$, but rather generated securely on-the-fly. It is merely used for padding with two random bits, and to pick the sign of the curve point's y-coordinate at random as suggested in \cite{elligatorExplicitFormulas}.

\begin{figure}
    \begin{pchstack}[boxed, center, space=0.5em]

    \begin{pcvstack}[space=0.05em]
    % obfs4 usage via common/ntor
    \procedure[linenumbering]{Initialization}{
        \pclinecomment{$\pk_\mathsf{id}$ is part of the bridge information:} \\
        (\pk_\mathsf{id},\sk_\mathsf{id}) \getsr \mathsf{NewKeypair}(\mathsf{ell2}=0) \\
        \pclinecomment{upon the start of a new handshake:} \\
        (\pk_c,\sk_c) \getsr \mathsf{NewKeypair}(\mathsf{ell2}=1) \\
        (\pk_s,\sk_s) \getsr \mathsf{NewKeypair}(\mathsf{ell2}=1)
    }

    \procedure[linenumbering]{$\mathsf{NewKeypair}(\mathsf{ell2} \in \bin)$}{
        \pcdo \\
        \t  r \getsr \bin^{256} \\
        \t  d \gets \operatorname{\mathsf{SHA2-512}}(r) \\
        \t  \sk \gets \mathsf{firstBytes}(d,32) \\
        \t  \pcif \mathsf{ell2} = 1 \pcthen \\
        \t  \t  t \gets \mathsf{lastBytes}(d,1) \\
        \t  \t  \pclinecomment{inlined ScalarBaseMult(\sk, t)} \\
        \t  \t  u \gets \mathsf{scalarBaseMultDirty}(\sk) \\
        \t  \t  \pk \gets \mathsf{Bytes}(u) \\
        \t  \t  \hat \pk \gets \mathsf{uToRepresentative}(u, t) \\
        \t  \t  \pcif \pk \neq \bot \pcthen \\
        \t  \t  \t  \pcreturn (\pk, \hat \pk, \sk) \\
        \t  \t  \pcendif \\
        \t  \pcelse \\
        \t  \t  \pk \gets \mathsf{curve25519}.\mathsf{ScalarBaseMult}(\sk) \\
        \t  \t  \pcreturn (\pk, \bot, \sk) \\
        \t  \pcendif \\
        \pcwhile \pctrue
    }
   
    \procedure[lnstart=4,linenumbering]{$\mathsf{ServerHandshake}(\hat \pk_c, (\pk_s, \sk_s), (\pk_\mathsf{id},\sk_\mathsf{id}))$}{
        \pk_c \gets \mathsf{RepresentativeToPublicKey}(\hat \pk_c) \\
        K_1 \gets \mathsf{curve25519}.\mathsf{ScalarMult}(\sk_s, \pk_c) \\
        K_2 \gets \mathsf{curve25519}.\mathsf{ScalarMult}(\sk_\mathsf{id}, \pk_c) \\
        \pcif K_1 = 0 \lor K_2 = 0 \pcthen \pcabort \pcendif
    }
    
    \procedure[lnstart=4,linenumbering]{$\mathsf{ClientHandshake}((\pk_c, \sk_c), \hat \pk_s, \pk_\mathsf{id})$}{
        \pk_s \gets \mathsf{RepresentativeToPublicKey}(\hat \pk_s) \\
        K_1 \gets \mathsf{curve25519}.\mathsf{ScalarMult}(\sk_c, \pk_s) \\
        K_2 \gets \mathsf{curve25519}.\mathsf{ScalarMult}(\sk_c, \pk_\mathsf{id}) \\
        \pcif K_1 = 0 \lor K_2 = 0 \pcthen \pcabort \pcendif
    }
    \end{pcvstack}

    \begin{pcvstack}[space=0.05em]
    % KEM components
    \procedure[linenumbering]{$\KEM.\kgen()$}{
        r \getsr \bin^{256} \\
        d \gets \operatorname{\mathsf{SHA2-512}}(r) \\
        \sk \gets \mathsf{firstBytes}(d,32) \\
        u \gets \mathsf{scalarBaseMultDirty}(\sk) \\
        \pk \gets \mathsf{Bytes}(u) \\
        \pcreturn (\pk, \sk)
    }
   
    \procedure[lnstart=4,linenumbering]{$\KEM.\encaps(\pk)$}{
        \pk_e, \sk_e \getsr \kgen() \\
        K \gets \mathsf{curve25519}.\mathsf{ScalarMult}(\sk_e, \pk) \\
        \pcif K = 0 \pcthen \\
        \t  \pcabort \\
        \pcendif \\
        \pcreturn (c=\pk_e, K)
    }
    
    \procedure[lnstart=9,linenumbering]{$\KEM.\decaps(\sk, c)$}{
        u \gets \mathsf{SetBytes}(c) \\
        K \gets \mathsf{curve25519}.\mathsf{ScalarMult}(\sk, \mathsf{Bytes}(u)) \\
        \pcif K = 0 \pcthen \\
        \t  \pcabort \\
        \pcendif \\
        \pcreturn K
    }

    % FEO components
    \procedure[linenumbering]{$E.\filterpk(\pk)$}{
        \pclinecomment{no restriction on $\pk$ to encapsulate against} \\
        \pcreturn \pctrue
    }
    
    \procedure[lnstart=3,linenumbering]{$E.\encodectxt(c)$}{
        t \getsr \bin^{8} \\
        u \gets \mathsf{SetBytes}(c) \\
        \hat \pk \gets \mathsf{uToRepresentative}(u, t) \\
        \pclinecomment{$\hat \pk$ is $\bot$ for 50\% of all values of $c$} \\
        \pcreturn \hat \pk
    }
    
    \procedure[lnstart=5,linenumbering]{$E.\decodectxt(\hat c)$}{
        \pcreturn \mathsf{RepresentativeToPublicKey}(\hat c)
    }
    \end{pcvstack}

\end{pchstack}
    \caption[
        Pseudocode illustrating the X25519 KEM and filter-encode obfuscator added to the \texttt{internal/x25519ell2} module.
    ]{
        Pseudocode illustrating the X25519 KEM $\KEM$ and a filter-encode obfuscator $E$, both added to the \texttt{internal/x25519ell2} module.
        The left-hand side of the figure shows the usage of the X25519 module by \obfsfour{}, while the right-hand side defines a KEM and obfuscator for use in \drivel{}.
        The aborts identify insecure secrets caused by low-order public keys.
        We use $\mathsf{firstBytes}(x, \ell), \mathsf{lastBytes}(x, \ell)$ to denote selecting the first or last $\ell$ bytes of $x$, respectively.
        $\mathsf{curve25519}$ denotes the submodule of \texttt{golang.org/x/crypto} by the same name, all other functions are from the lyrebird repository's \texttt{internal/x25519ell2} and \texttt{common/ntor} modules.
    }
    \label{fig:impl-x25519-kem}
\end{figure}

\paragraph{Adding Crypto-Agility to the \texttt{internal} Module}

Observing the current need for post-quantum key exchange, combined with the increasing deployment of newly standardized algorithms, it seems prudent to implement flexibility in the utilized key exchange schemes. Being able to exchange cryptographic algorithms is often called ``being crypto-agile''.

To this end, the \drivel{} implementation contains string constants identifying the desired KEM and OKEM. The data structures for the schemes are then loaded at runtime based on these strings.
The schemes are identified by the following naming convention:
\begin{itemize}
    \item The string "x25519" refers to the unobfuscated, newly implemented KEM from \cref{fig:impl-x25519-kem},
    \item all other strings denote \texttt{liboqs} KEMs, and
    \item any KEM with a corresponding filter-encode obfuscator can be used to request an OKEM under the KEM name prefixed with ``FEO-'', e.g.~``FEO-x25519'' would denote $\feo[\KEM, E]$ using the components from \cref{fig:impl-x25519-kem}. This includes ML-Kemeleon, which we name as ``FEO-ML-KEM''.
\end{itemize}

Note that some ``FEO-'' schemes may in fact simply be the unmodified KEM (the code uses a \texttt{nil} encoder reference), but the naming convention ensures that the name of a scheme clearly separates KEMs from OKEMs. Additionally, as discussed above in \cref{sec:parameters}, some KEMs and OKEMs available from \texttt{liboqs} are disabled to limit the number of test cases and enable some optimizations with regards to buffer sizes.

Due to the different origin of these KEM and OKEM implementations, and the lack of an abstract standardized KEM interface in Go suitable for these constructions, we added several new submodules to \texttt{internal}:
\begin{itemize}
    \item \texttt{internal/cryptodata} offers a \texttt{CryptoData} type which holds a byte string. Conversions to and from hexadecimal representation are implemented here, as well as assertions checking the size of these byte strings. The latter is enforced as part of the constructor. This helps to prevent accidental changes to the size of these objects, and ensures that we are always working with fixed-length strings. Internally, these objects are simply Go byte slices but our type restricts the possible operations to a minimum.

    \item \texttt{internal/kems} and \texttt{internal/okems} are two submodules that implement KEM and OKEM interfaces, as well as data types for serializable public keys, private keys, keypairs, ciphertexts and shared secrets. While these submodules contain very similar code, the differentiation ensures that e.g. KEM ciphertexts can only be used in a context where OKEM ciphertexts are required by explicitly casting from one type to another. Again, this helps the readability and prevents certain implementation mistakes.

    \item \texttt{internal/cryptofactory} contains the final additional submodule and constitutes our main contribution to the \texttt{internal} module. Its job is to expose a list of supported KEMs and OKEMs for test purposes, as well as two constructors \texttt{NewKem} and \texttt{NewOkem} allowing for the construction of a scheme based only on its name as described above. Several nested submodules implement specific functionality:
    \begin{itemize}
        \item \texttt{internal/cryptofactory/encoding\_classic\_mceliece} implements a filter-encode-obfuscator for ``Classic-McEliece-6960119'',
        \item \texttt{internal/cryptofactory/encoding\_hqc} implements filter-encode-obfuscators for all three HQC parameter sets,
        \item \texttt{internal/cryptofactory/encoding\_kemeleon} implements the ``Non-Rejection Sampling Variant'' of the filter-encode-obfuscators for ML-KEM from \cite{irtf-cfrg-kemeleon-00},
        \item \texttt{internal/cryptofactory/filter\_encode} defines an interface for these obfuscators and implements the generic FEO transformation from \cref{def:filter-encode-okem}, and, finally,
        \item \texttt{internal/cryptofactory/oqs\_wrapper} implements a thin wrapper which makes the \texttt{liboqs} KEM interface compatible with our interface defined in \texttt{internal/kems}\footnote{Our interface differs from the existing \texttt{liboqs} and Go interfaces in that it uses the data types introduced in \texttt{internal/cryptodata}, more closely matches the KEM syntax used in this thesis, and provides fewer pieces of information about the KEM.}.
    \end{itemize}
\end{itemize}

Each of the filter-encode obfuscators is covered with unit tests to check that ciphertexts are recovered without change after encoding and subsequent decoding. Our theoretical first-keygen and first-encaps success probabilities are verified, and some sanity-checks are performed to validate the encoding outputs. All available KEMs and OKEMs are also tested for their correctness and benchmarked for their performance as part of the \texttt{internal/cryptofactory} submodule.

\paragraph{Notable Differences between \texttt{transports/drivel} and \texttt{transports/obfs4}}

Although the \texttt{transports/drivel} module started out as a copy of \texttt{transports/obfs4}, multiple conceptual changes were made during the course of this thesis.

While \texttt{transports/obfs4} depends on \texttt{common/ntor} to implement the main cryptography beyond obfuscation, this functionality is now provided instead by \texttt{transports/drivel/drivelcrypto}. This functionality is now less spread out through the repository. Test cases from the \texttt{common/ntor} module were copied and extended as needed.
Functionality of this new submodule includes:
\begin{itemize}
    \item the generation and (de-)serialization of $\mathsf{NodeID}$,
    \item the definition of types for the values $\mathsf{auth}$ and $\mathsf{KEY\_SEED}$,
    \item constant-time comparisons for $\mathsf{MAC_C}, \mathsf{MAC_S}$ and $\mathsf{auth}$,
    \item the derivation of $\mathsf{auth}$ and $\mathsf{KEY\_SEED}$ after all protocol messages have been parsed and authenticated (as this functionality is shared between client and server),
    \item functions to calculate $\mathsf{M_C}, \mathsf{M_S}, \mathsf{MAC_C}, \mathsf{MAC_S}$ more easily,
    \item all relevant string constants for these procedures,
    \item implementations for $F_1, F_2$ and the symmetric encryption scheme $SE$
\end{itemize}

The framing layer \texttt{transports/drivel/framing} and packet layer \texttt{transports/drivel/packet.go} are essentially unchanged compared to \texttt{transports/obfs4}.

The main protocol functionality is implemented in \texttt{transports/drivel/handshake.go}, handling state objects persisted through the course of a connection, and implementing functions to initialize this state, parse messages, and generate responses.
Sitting between the low-level cryptography primitives and the higher-level handshake process, this file was significantly adapted from its predecessor \texttt{transports/obfs4/handshake\_ntor.go}, and now employs the newly added data types from \texttt{internal}.

One noteworthy change in particular is that the sizes of public keys, private keys, ciphertexts and shared secrets is not statically known anymore, as the schemes are only selected at runtime.
Instead of hardcoding sizes, we respect sizes reported by \texttt{liboqs} at runtime to further reduce the changes required to switch out a KEM or OKEM.
This approach led to the addition of some runtime calculations of possible padding values and expected sizes of message components in a new \texttt{lengthDetails} object.
This does, however, enable us to provide a single binary to clients, which is capable (across multiple sessions) of connecting to different bridges, each of which may use a different selection of KEM and OKEM schemes. Thus, clients retain flexibility and interoperability with many bridges while bridge operators can control the employed KEM/OKEM combination.

As a final change, we consider \texttt{transports/drivel/drivel.go} and \texttt{transports/drivel/statefile.go} which handle the high-level protocol functionality such as parsing arguments, and loading saved bridge information, as well as constructing the connection used for later data transfer after the handshake phase is completed.
A significant change here concerned the passing of bridge information between the client's configuration file and the pluggable transport implementation: The Tor ecosystem requires all pluggable transports to implement a SOCKS proxy setup. The content of the client's configuration is then sent to its local SOCKS proxy via the username and password fields (as the proxy is only available locally, there is no actual authentication) and the bridge IP is provided as a target address. In \texttt{transports/obfs4}, this configuration contains the $\mathsf{NodeID}$ as well as the server's static public key $\pk_S$. However, for post-quantum OKEMs the size of this public key exceeds the limit of SOCKS username and password sizes, which may have a combined length of at most 510 bytes. This issue has already been noted by David Fifield in the context of the Snowflake pluggable transport \cite{torprojectSnowflakeBridge}. While possible solutions have been discussed in the Tor Project, there is (at the time of writing) no standardized solution as part of the Tor ecosystem.

To circumvent this limitation, we require clients to:
\begin{itemize}
    \item add the so-called ``bridge-line'' to their configuration file (as for \obfsfour{}), containing (among other things) the name of the pluggable transport, IP and port of the bridge, and the $\mathsf{NodeID}$, but now also
    \item download a public key file called \texttt{drivel\_key-<KEYID>} containing $\pk_S$ in JSON format, where \texttt{<KEYID>} is a prefix of $\mathsf{NodeID}$. This file is created in the correct format by the bridge upon starting.
\end{itemize}

As the configuration is now considerably shorter, it can be transmitted via the SOCKS fields, and the pluggable transport can load the public key corresponding to $\mathsf{NodeID}$ from the disk.

Additionally, the public key file contains the names of the KEM and OKEM schemes that the bridge is configured to use. This ensures interoperability and allows clients to use different schemes depending on the selected bridge.

\section{Deployment Guide} \label{sec:deployment}

Note that this setup procedure is not geared towards end users, and a more user-friendly procedure is required for wide-spread deployment.
However, since this involves multiple changes to the current distribution mechanisms and \drivel{} may be further adapted before then, we believe that further improvements are premature at this point.

In order to deploy our implementation of \drivel{} as a Tor pluggable transport, we recommend the following procedure:
\begin{enumerate}
    \item Clone our fork of the lyrebird repository\footnote{See \url{https://gitlab.ethz.ch/himarc/lyrebird-pq-obfs}} into a new directory called ``lyrebird''
    
    \item Clone the official Tor repository containing recommended container files from the ``docker-obfs4-bridge'' repository\footnote{See \url{https://gitlab.torproject.org/tpo/anti-censorship/docker-obfs4-bridge}} into another directory called ``docker'' next to the one just created
    
    \item Adapt the files of ``docker-obfs4-bridge'' to match the contents of the corresponding files in the \texttt{docker/bridge} folder in our analysis repository\footnote{See \url{https://gitlab.ethz.ch/himarc/thesis-quantumsafe-fep}}. This changes the build process to not use the official lyrebird repository but instead copy the contents of a local ``lyrebird'' folder, install all necessary build dependencies and perform the build. Additionally, the bridge-line format requires minor changes to the ``get-bridge-line'' and ``start-tor.sh'' files.

    \item For production use, perform the following additional steps:
    \begin{enumerate}
        \item Remove the file \texttt{wrapper.sh}, and delete the following lines from the file \texttt{Dockerfile}
        \begin{lstlisting}
COPY docker/bridge/wrapper.sh /usr/bin/wrapper.sh
RUN chmod +x /usr/bin/wrapper.sh
        \end{lstlisting}

        \item Replace ``/usr/bin/wrapper.sh'' by ``/usr/bin/lyrebird'' in \texttt{start-tor.sh}

        \item Re-enable the publication of server descriptors and the bridge distribution by removing the following lines from \texttt{start-tor.sh}:
        \begin{lstlisting}
PublishServerDescriptor 0
BridgeDistribution none
        \end{lstlisting}
    \end{enumerate}
    \label{bul:prod-use-changes}

    \item The rest of the setup requires merely setting the correct environment variables and building the docker container. This is well described in the official Tor guide\footnote{See \url{https://community.torproject.org/relay/setup/bridge/docker/}}, which references a \texttt{docker-compose.yml} file. Simply download this file, place it next to the two folders created above and replace the line ``image: thetorproject/obfs4-bridge:latest'' by the following three lines:
    \begin{lstlisting}
build:
  context: .
  dockerfile: docker/bridge/Dockerfile
    \end{lstlisting}

    \item Additionally, make sure that you set the environment variable ``TOR\_PT\_SERVER\_TRANSPORT\_OPTIONS'' in the \texttt{docker-compose.yml} file. Specify the variables ``kem-name'' and ``okem-name'' to configure the schemes used by \drivel{}. An example is:
    \begin{lstlisting}
TOR_PT_SERVER_TRANSPORT_OPTIONS="drivel:kem-name=ML-KEM-512;drivel:okem-name=FEO-Classic-McEliece-348864"
    \end{lstlisting}

    \item Finally, start the tor bridge using the command \texttt{docker-compose up -d} in the same directory as \texttt{docker-compose.yml}
\end{enumerate}

To connect, clients then require the information output by ``get-bridge-line'' in the running docker container, and all JSON files in the folder \texttt{/var/lib/tor/pt\_state/} whose name starts with \texttt{drivel\_key-}.

For clients, the same procedure to compile the lyrebird repository can be reused. Additionally, an installation of Tor is required, the bridge-line should be added to the \texttt{torrc} file and the public key file should be placed into the folder \texttt{/var/lib/tor/pt\_state}. This procedure is simpler as the format of bridge-lines does not have to be accounted for, and the main change is the use of a locally compiled lyrebird binary.

An example of such a setup can be seen in our analysis repository, which is configured to run both a bridge and a client container locally in a very similar manner, albeit with more debugging options and automatic transfer of the bridge-line and public keys, thus the changes described above in \cref{bul:prod-use-changes}.

\section{Challenges and Lessons Learned} \label{sec:challenges-learnings}

To summarize, the implementation of \drivel{} to the point of working test deployments, reusing existing tools for \obfsfour{} was a success.

Some modifications to the protocol may prove promising in the future, both in terms of flexibility and performance, but also for obfuscation. Nonetheless, we believe this implementation to provide a valuable starting point for future work.

In the following, we list some concrete implementation challenges experienced during this thesis:
\begin{itemize}
    \item KEM public keys were too large to use the XOR symmetric encryption as outlined in \cite{EPRINT:GRSV25}, as the HKDF could not provide enough key material. This required the use of AES-CTR.

    \item KEM public keys were also too large to send as part of the bridge-line in the SOCKS fields. We therefore added public key files, which are generated by the bridge and must also be sent to the client.

    \item Due to the crypto-agility introduced, the client now needs to know ahead of time, what KEM and OKEM scheme the server is configured to use. A negotiation as part of the handshake was out-of-scope for this thesis. Thus, we made servers schemes configurable via the ``TOR\_PT\_SERVER\_TRANSPORT\_OPTIONS'' environment variable and saved the names of schemes (following our naming convention) into the public key file distributed alongside the bridge-line. Clients, upon receiving $\mathsf{NodeID}$ via the SOCKS fields then load the correct file, and are now able to initialize the KEM and OKEM accordingly.

    \item This thesis contains one of the first implementations of ML-Kemeleon. One previous implementation \cite{ct-kemeleon} did not implement the \texttt{SamplePreimage} function from the Internet-Draft, and another \cite{jmwample-kemeleon} only had partial test cases for this function, and did not correctly sample uniformly random outputs. We implemented comprehensive tests, and verified the expected output distribution of \texttt{SamplePreimage}. Necessary changes to the Internet-Draft were identified in the process, and the draft authors, as well as the authors of the two previous implementations were informed of our findings.

    \item An unaddressed issue is the need to hash certain pieces of data multiple times over the course of the protocol due to the design of \drivel{}. In fact, our implementation does suffer from high memory usage compared to \obfsfour{}, and optimizations to reduce that would be beneficial. As the keys and suffixes used in HKDF change, the previous results cannot easily be reused for future calculations. In particular, requesting multiple output blocks from $F_1$ would quickly become quite costly. We consequently chose our output lengths appropriately to avoid this effect and suggest an alternative approach further below in \cref{sssec:variant-framing}.
\end{itemize}

In order for software to remain flexible for future adaptations, it is beneficial to avoid hardcoding specific algorithms. A modular approach where components such as encryption, PRFs, and KEMs can be easily exchanged ensures that future developments cause fewer changes to the overall system. Even apparently harmless limitations such as SOCKS field lengths can constitute integration barriers in the future.

As a final note, the highly transparent and open-source development of the Tor project enabled us to focus directly on making material changes to an existing implementation. The open ecosystem was extremely valuable and allowed us to contribute more easily.

