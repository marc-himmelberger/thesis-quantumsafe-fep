\chapter{Implementing drivel}\label{ch:implementation}

Use Kemeleon alongside the other obfuscated KEMs

\section{The Drivel Protocol} \label{sec:drivel}

\section{Implementation} \label{sec:implementation}

* (Experimenting with implementing different traffic patterns, such as arbitrary fragmentation of key
exchange messages and/or more flexibility in clients to send arbitrary data)


=> For Tor people: Consider as part of audience of report, so also accessible to non-crypto-nerds and illustrate: How to deploy, How to use, Integration barriers such as SOCKS issues -> Implementation section


Fragmentation:
* Drivel protects against retroactive identification of traffic possible in obfs4 by involving K_S
    Using K_S requires some minimum amount of traffic before a response is sent from the server (ctxt + PRF + padding)
    You could then do arbitrary fragmentation on top of K_S
* But if we want arbitrary fragmentation, you'd still need at least a PRF value; but could allow more traffic patterns - at the risk of having your traffic be identified...
-> seems to me like a weird mix of "Drivel is better than obfs4 because you can't identify it later" and then backtracking for fragmentation.

Minimum size would depend on the OKEM, and might be small e.g. for CM.
