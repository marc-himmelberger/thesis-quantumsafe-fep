\chapter{Preliminaries}\label{ch:preliminaries}

\paragraph{Notation}
In pseudocode, we use the notation $x \gets A(y)$ to express an assignment to a variable $x$ using the result of the deterministic computation $A(y)$. When the computation is probabilistic, we write $x \getsr A(y)$.
Uniformly random sampling of $x$ from a set $S$ is denoted $x \getsr S$.

For the definition of pseudorandom functions (PRF), message authentication codes (MAC), authenticated encryption (AE), symmetric encryption (SE), as well as the notion of PRF security, we refer readers to the introductory textbook \cite{katz_lindell}.
Dual-PRF security is defined in \cite[Section~7.1]{CCS:GunSteVei24}.

\section{Review of Prior Work} \label{sec:review-gsv24}

Günther et al.~\cite{CCS:GunSteVei24} first described a framework for the provable security of obfuscated key exchange, extending efforts by Fenske and Johnson \cite{CCS:FenJoh24}, whose work focused on the data transfer phase of fully encrypted protocols.

The work of Günther et al.~was expanded upon in \cite{EPRINT:GRSV25} where the notation was simplified and the proposed protocol from \cite{CCS:GunSteVei24} was slightly modified to yield \drivel{}, the obfuscated key exchange protocol we will implement in this thesis.

We adapt the definitions from \cite{CCS:GunSteVei24,EPRINT:GRSV25} to our setting. Concretely, we omit the concept of obfuscated public keys and focus only on ciphertext obfuscation, as public keys are only transmitted in encrypted form (obviating the need for public-key obfsucation) or out-of-band in \drivel{}. Note that we will retain the ability to reject certain public keys, as this can be required for obfuscating ciphertexts. The definitions are equivalent in all other aspects.

\subsection{Key Encapsulation Mechanisms}

\begin{definition}[Key encapsulation mechanism]
    \label{def:kem}
    A key encapsulation mechanism $\KEM = (\kgen, \encaps, \decaps)$ consists of three algorithms:
    \begin{itemize}
        \item $\ul{\kgen}() \tor (\pk, \sk)$
        is a probabilistic \emph{key generation} algorithm that generates a public key~$\pk$ and corresponding secret key~$\sk$.
        \item $\ul{\encaps}(\pk) \tor (c, K)$
        is a probabilistic \emph{encapsulation} algorithm that takes as input a KEM public key~$\pk$, and outputs a ciphertext~$c$ and shared secret~$K$.
        \item $\ul{\decaps}(\sk, c) \to K$
        is a deterministic \emph{decapsulation} algorithm that takes as input a secret key~$\sk$ and ciphertext~$c$, and outputs a shared secret~$K$.
    \end{itemize}
\end{definition}

\begin{definition}[KEM correctness]
\label{def:kem-corr}
We say that a KEM~$\KEM=(\kgen, \encaps, \decaps)$ is \emph{$\delta_\KEM$-correct} if
\[
    \Pr\left[
        \decaps(\sk,c) \neq K
    ~\middle|
        \begin{array}{c}
        (\pk,\sk) \getsr \kgen(), \\
        (c, K) \getsr \encaps(\pk)
        \end{array}
    \right] \leq \delta_\KEM.
\]
\end{definition}

To define what it means for a KEM to be secure, we require two security notions. One focuses on indistinguishability of ciphertexts ($\indcca$), i.e., given a public key and ciphertext, it should be hard to differentiate the encapsulated shared secret from uniformly random elements of the key space.
This captures the goal that an adversary (given the public key $\pk$) does not learn any information about the shared secret encapsulated in the challenge ciphertext $c^*$, even when allowed to obtain decapsulations of other ciphertexts.
This notion is central to the main premise of KEMs, namely to initialize a secure communication channel given only one party's public key by means of setting up a shared secret.

A second notion is strong pseudorandomness ($\sprcca$), i.e., given a public key, it should be hard to differentiate a ciphertext-key pair obtained from encapsulation for that public key from a pair consisting of a simulated ciphertext and a uniformly randomly sampled key, even when allowed to obtain decapsulations of other ciphertexts.
This captures the goal that ciphertexts must "look randomly sampled" among some set of possible ciphertexts and should look unrelated to any given public key --- even given the corresponding shared secret.
This is expressed through the use of a simulator $\mathcal S$ and $\sprcca$ is always achieved with respect to such a simulator.
This notion was first introduced in this form by Xagawa \cite{EC:Xagawa22} in the context of establishing anonymity of KEMs, and is satisfied by many KEMs that are being considered for use in quantum-safe communication \cite{EC:Xagawa22}. We use $\sprcca$ security to argue about the distribution of ciphertexts.

\begin{definition}[KEM security] \label{def:kem-security}
    We define two security notions, $\indcca$ and $\sprcca$, using the games shown in \cref{fig:kem-security}, and we define the respective advantages of an adversary~$\adv$ against the $\mathsf{goal} \in \{\indcca, \sprcca\}$ security of a KEM~$\KEM$ as
    \[
        \advantage{\mathsf{goal}}{\KEM(, \mathcal{S})}[(\adv)] := 2 \cdot \left| \Pr \left[ G^{\mathsf{goal}}_{\KEM(, \mathcal{S})}(\adv) \Rightarrow 1 \right] - \frac{1}{2} \right|,
    \]
    where the simulator $\mathcal{S}$ is only included for $\sprcca$ security.

    We also define the weaker versions of $\indcpa,\sprcpa$ security as the same games but with the restriction that adversaries may not make any queries to their decapsulation oracle $\decapsoracle$.

    We also define two common choices for the simulators used in the game $\sprcca$: \begin{itemize}
        \item Let $\simulunif$ be a simulator that outputs uniformly random ciphertexts from the ciphertext space of $\KEM$.
        \item Let $\simulbin$ be a simulator that outputs uniformly random elements of $\bin^\ell$, where $\ell$ is a suitable integer representing the length of ciphertexts in bits.
    \end{itemize}
\end{definition}

\begin{figure}
    \begin{pchstack}[boxed, center, space=0.5em]

\begin{pcvstack}[space=0.05em]
\procedure[linenumbering]{$\textbf{GAME } \indcca(\adv)$}{
    b \getsr \bin \\
    (\pk, \sk) \getsr \kgen() \\
    (c^*, K_0) \getsr \encaps(\pk) \\
    K_1 \getsr \mathcal K \\
    b' \getsr \adv^{\decapsoracle}(\pk, c^*, K_b) \\
    \pcreturn \llbracket b = b'\rrbracket
}

\procedure[lnstart=6,linenumbering]{$\decapsoracle(c \neq c^*)$}{
    K \gets \decaps(c) \\
    \pcreturn K
}

\end{pcvstack}

\begin{pcvstack}[space=0.05em]
\procedure[lnstart=8,linenumbering]{$\textbf{GAME } \sprcca(\adv)$}{
    b \getsr \bin \\
    (\pk, \sk) \getsr \kgen() \\
    (c_0, K_0) \getsr \encaps(\pk) \\
    c_1 \getsr \mathcal S \\
    K_1 \getsr \mathcal K \\
    c^* \gets c_b \\
    b' \gets \adv^{\decapsoracle}(\pk, c_b, K_b) \\
    \pcreturn \llbracket b = b'\rrbracket
}
\end{pcvstack}

\end{pchstack}

    \caption[
        Security games for $\indcca$ and $\sprcca$ security of a KEM or obfuscated KEM.
    ]{
        Security games for $\indcca$ and $\sprcca$ security of a KEM~$\KEM = (\kgen, \encaps, \decaps)$ with key space~$\mathcal K$. $\sprcca$ security is defined with respect to a simulator $\mathcal S$. Oracle queries violating the stated condition on arguments are not allowed.
    }
    \label{fig:kem-security}
\end{figure}

\begin{definition}[KEM public key collision probability]
    \label{def:pk-collisions}
    Let $\KEM$ be a KEM.
    We define the \emph{public key collision probability} of $\KEM$ for $n \in \mathbb{N}$ public keys as
    \[
        \pkcoll{\KEM}(n) := \Pr\left[
            \begin{array}{c}
                \pk_i = \pk_j \\
                \land~ i \neq j
            \end{array}
            \middle|
            \begin{array}{c}
                (\pk_i,\sk_i) \getsr \KEM.\kgen() \\
                \text{ for } i \in [1,n]
            \end{array}
            \right].
    \]
\end{definition}

\subsection{Obfuscated KEMs}

\begin{definition}[Obfuscated KEM]
    \label{def:okem}
    An \emph{obfuscated key encapsulation mechanism (OKEM)} $\OKEM = (\kgen, \encaps, \decaps)$ with \emph{obfuscated ciphertext length}~$\obfctxtlen \in \mathbb{N}$ consists of the following algorithms:
    \begin{itemize}
        \item $\ul{\kgen}() \tor (\pk,\sk)$ is a probabilistic \emph{key generation} algorithm that generates a public key~$\pk$ and corresponding secret key~$\sk$.

        \item $\ul{\encaps}(\pk) \tor (\hat c, K)$ is a probabilistic \emph{encapsulation} algorithm that takes as input an OKEM public key~$\pk$ and outputs an (obfuscated) ciphertext~$\hat c \in \bin^\obfctxtlen$ and shared secret~$K$.

        \item $\ul{\decaps}(\sk,\hat c) \to K$ is a deterministic \emph{decapsulation} algorithm that takes as input a secret key~$\sk$ and (obfuscated) ciphertext~$\hat c$, and outputs a shared secret~$K$.
    \end{itemize}
    Note that any OKEM, as defined here, can also be viewed as a KEM as in \cref{def:kem}. Crucially, however, the ciphertext space of an OKEM must a a space of fixed-length bit strings.

    As in \cref{def:kem-corr}, we also demand the KEM correctness for OKEMs.
\end{definition}

For OKEMs, the notions of $\indcca$ security, OKEM correctness, and public-key collision probability are defined exactly as for KEMs (see \cref{fig:kem-security,def:kem-corr,def:pk-collisions}).

We occasionally denote the length in bits of the KEM public keys, KEM ciphertexts, OKEM public keys, and OKEM ciphertexts as $\pklen, \ctxtlen, \obfpklen, \obfctxtlen$, respectively.

In addition to the above properties, an obfuscated KEM should satisfy additional randomness requirements. In particular, its ciphertexts should not be distinguishable from random bit strings. This is in contrast to regular KEMs that may have recognizable structure in $c$ even if they are $\sprcca$-secure.
For this reason, we also restrict the more general definition of $\sprcca$ security for KEMs to always use the simulator $\simulbin$ with $\ell = \obfctxtlen$ in the context of OKEMs.

\begin{definition}[Ciphertext uniformity]
\label{def:ctxt-uniformity}
    Let $\OKEM$ be an OKEM.
    We measure the \emph{ciphertext uniformity} of the obfuscated ciphertext of length~$\obfctxtlen$ generated by $\OKEM.\encaps$ against an adversary~$\adv$ as
    \[
        \advantage{\ctxtunif}{\OKEM}[(\adv)] := 
        2 \cdot \left|
        \Pr\left[
            \adv(\pk, \hat c_b) = b
        ~\middle|
            \begin{array}{c}
                b \getsr \bin, \hat c_0 \getsr \bin^{\obfctxtlen}, \\
                (\pk,\sk) \getsr \OKEM.\kgen(),\\
                (\hat c_1, K_1) \getsr \OKEM.\encaps(\pk)
            \end{array}
        \right]
        - \frac{1}{2}
        \right|.
    \]
    
    For an unbounded adversary~$\adv$, we call the advantage~$\advantage{\ctxtunif}{\OKEM}[(\adv)] $ \emph{statistical}.
\end{definition}

\subsection{Constructing OKEMs}

To construct an OKEM, a regular KEM can be augmented with encoding functions similar to \textsf{Elligator2}~\cite{CCS:BHKL13} or \textsf{Kemeleon}~\cite[Sec.~2.4]{CCS:GunSteVei24}. We define the following construct to bundle such functions --- note, however, that we do not require encoded public keys in this work, which is why we will not aim to capture all components of \textsf{Elligator2} or \textsf{Kemeleon}.

The construction below relies on two primary components to provide obfuscation of ciphertexts: First, the construction may reject or ``filter'' certain KEM public keys to ensure that the public keys and ciphertexts appear unrelated. Secondly, the ciphertexts generated by the KEM's encapsulation function might be ``encoded'' to resemble uniformly random bit strings and thus hide their structure. Together, these components are capable of obfuscating KEM ciphertexts.

\begin{definition}[Filter-Encode Obfuscator]
\label{def:filter-encode-obfs}
    Let $\KEM = (\kgen, \encaps, \decaps)$ be a KEM.
    Let $\obfctxtlen \in \mathbb N$.
    We then define a \emph{Filter-Encode Obfuscator} $E$ for $\KEM$ to consist of the following efficient algorithms operating on the appropriate input/output spaces of $\KEM$:
    \begin{itemize}
        \item $\ul{\filterpk}(\pk) \to \bin$
        is a deterministic \emph{public key filtering} algorithm that, on input of a public key~$\pk$ previously generated by $\KEM.\kgen$, outputs a single bit, denoting whether the public key should be kept (value $1$) or rejected (value $0$).
        \item $\ul{\encodectxt}(c) \tor \hat c$
        is a (possibly randomized) \emph{ciphertext encoding} algorithm that, on input of a ciphertext~$c$ from the ciphertext space of $\KEM$, outputs an obfuscated ciphertext~$\hat c \in \bin^{\obfctxtlen}$ or an error $\bot$.
        \item $\ul{\decodectxt}(\hat c) \to c$
        is a deterministic \emph{ciphertext decoding} algorithm that, on input of an obfuscated ciphertext~$\hat c \in \bin^{\obfctxtlen}$, outputs a ciphertext~$c$ from the ciphertext space of $\KEM$.
    \end{itemize}

    We demand that these encoding/decoding algorithms are perfectly correct, i.e.:
    \[
        \Pr\left[
            \decodectxt(\hat c) = c
        ~\middle|
            \begin{array}{c}
                (\pk, \sk) \getsr \kgen(),\\
                \filterpk(\pk) = 1, \\
                (c, K) \getsr \encaps(\pk),\\
                \hat c \getsr \encodectxt(c),\\
                \hat c \neq \bot
            \end{array}
        \right] = 1.
    \]
\end{definition}

\begin{definition}[First-keygen Success Probability]
\label{def:first-keygen-success}
    Let $\KEM = (\kgen, \encaps, \decaps)$ be a KEM.
    Let $E = (\filterpk, \encodectxt, \decodectxt)$ be a filter-encode obfuscator for $\KEM$.
    We then define the \emph{first-keygen success probability} as the probability that a randomly generated public key is not rejected, i.e.
    \[
        \epsilon^\firstkeygensuccess_{\KEM,E} :=
        \Pr\left[
            \filterpk(\pk) = 1
        ~\middle|
            \begin{array}{c}
                (\pk, \sk) \getsr \kgen()
            \end{array}
        \right].
    \]
\end{definition}

\begin{definition}[First-encaps Success Probability]
\label{def:first-encaps-success}
    Let $\KEM = (\kgen, \encaps, \decaps)$ be a KEM.
    Let $E = (\filterpk, \encodectxt, \decodectxt)$ be a filter-encode obfuscator for $\KEM$.
    We then define the \emph{first-encaps success probability} as the probability that a randomly generated ciphertext is not rejected during encoding, i.e.
    \[
        \epsilon^\firstencapssuccess_{\KEM,E} :=
        \Pr\left[
            \encodectxt(c) \neq \bot
        ~\middle|
            \begin{array}{c}
                (\pk, \sk) \getsr \kgen(), \\
                \filterpk(\pk) = 1, \\
                c \getsr \encaps(\pk)
            \end{array}
        \right].
    \]
\end{definition}

\begin{definition}[Filter-Encode Transformation]
\label{def:filter-encode-okem}
    Let $\KEM = (\kgen, \encaps, \decaps)$ be a KEM.
    Let $E = (\filterpk, \encodectxt, \decodectxt)$ be a filter-encode obfuscator for $\KEM$.
    We then define the corresponding \emph{filter-encode} obfuscated KEM $\feo[\KEM, E] := (\kgen', \encaps', \decaps')$ with the following new algorithms:

    \begin{pchstack}[boxed, center, space=0.5em]

    \procedure[linenumbering]{$\kgen'()$}{
        \pcdo \\
        \t  (\pk, \sk) \getsr \KEM.\kgen() \\
        \label{ln:kgen-encode}
        \t  \pkobf \getsr \encode(\pk) \\
        \pcwhile \pkobf = \bot \\
        \pcreturn (\pkobf, \sk)
    }
    
    \procedure[lnstart=5,linenumbering]{$\encaps'(\pkobf)$}{
        \pk \gets \decode(\pkobf) \\
        \pcdo \\
        \t  (c, K) \getsr \KEM.\encaps(\pk) \\
        \label{ln:encaps-encode}
        \t  \hat c \getsr \encodectxt(c) \\
        \pcwhile \hat c = \bot \\
        \pcreturn (\hat c, K)
    }
    
    \procedure[lnstart=11,linenumbering]{$\decaps'(\sk, \hat c)$}{
        c \gets \decodectxt(\hat c) \\
        \pcreturn \KEM.\decaps(c)
    }

\end{pchstack}
\end{definition}

Note that in this construction, the first-keygen success probability from \cref{def:first-keygen-success} is precisely the probability that \cref{ln:feo-reject-keygen} was executed only once in a given execution of $\kgen'$, and similarly, the first-encaps success probability from \cref{def:first-encaps-success} is precisely the probability that \cref{ln:feo-reject-ctxt} was executed only once in a given execution of $\encaps'$.

For an OKEM constructed according to \cref{def:filter-encode-okem} using appropriate encoding/decoding functions, it was shown in \cite[Theorems 2.12 and 2.13]{CCS:GunSteVei24} that the central security notions are preserved. The proofs both employ reductions to KEM security notions that reproduce the obfuscator's operations, aborting if public keys are rejected or encoding fails.
The results apply to our filter-encode transformation, as it can be viewed as a special case of the more general keygen/encaps-then-encode construction from \cite{CCS:GunSteVei24}:

\begin{theorem}[Filter-Encode Security]
\label{thm:filter-encode-security}
    Let $\OKEM = \feo[\KEM, E]$ be a filter-encode obfuscated obfuscated KEM based on a regular KEM $\KEM$ and a corresponding obfuscator $E$ as per \cref{def:filter-encode-obfs}.
    For any adversary $\adv$ against the $\indcca$ security of $\OKEM$, there exists an algorithm $\bdv$ such that
    \[
        \advantage{\indcca}{\OKEM}[(\adv)]
        \leq
        1/\epsilon^\firstkeygensuccess_{\KEM,E}
        \cdot 1/\epsilon^\firstencapssuccess_{\KEM,E}
        \cdot \advantage{\indcca}{\KEM}[(\bdv)].
    \]

    Further, for any adversary $\adv$ against the $\sprcca$ security of $\OKEM$, there exist algorithms $\bdv$, $\cdv$ such that
    \[
        \advantage{\sprcca}{\OKEM}[(\adv)]
        \leq 
        1/\epsilon^\firstencapssuccess_{\KEM,E}
        \cdot \advantage{\sprcca}{\KEM,\simulunif}[(\bdv)]
        + \advantage{\ctxtunif}{\KEM,E}[(\cdv)].
    \]

    Finally, the KEM correctness of $\OKEM$ is trivially preserved, as we require perfect correctness from $\encodectxt,\decodectxt$ in \cref{def:filter-encode-obfs}, i.e.
    \[
        \delta_\OKEM = \delta_\KEM.
    \]
\end{theorem}

Finally, the ciphertext uniformity of the resulting OKEM may be analyzed directly on the output distribution of the $\encodectxt$ algorithm, conditioned on the public key not being rejected. The following case is a generalization of the proof of \cite[Lemma~2.15]{CCS:GunSteVei24}, and we will use it to simplify later proofs.

\begin{lemma}[Ciphertext Uniformity from SPR-CPA]
\label{lem:ctxt-unif-for-bijections}
    Let $\OKEM = \feo[\KEM, E]$ be a filter-encode obfuscated KEM based on a regular KEM $\KEM$ and a corresponding obfuscator $E$ with a \emph{deterministic} encoding function $\encodectxt$.

    Let $\mathbf{P}' \subseteq \mathbf{P}$ denote the subset of the public key space of $\KEM$ for which $\filterpk$ outputs $1$, i.e. the set of allowed public keys.
    Let $\mathbf{C}' \subseteq \mathbf{C}$ then denote the subset of the ciphertext space of $\KEM$ where $\encodectxt$ does not output $\bot$, i.e. the set of encodeable ciphertexts.
    Let $\bin^\ell$ be the output space of $\encodectxt$.

    If $\encodectxt$ is a bijective function from $\mathbf{C}'$ to $\bin^\ell$, then for any adversary against the ciphertext uniformity of $\OKEM$, there exists an adversary $\bdv$ against the $\sprcpa$ security of $\KEM$ with respect to the simulator $\simulunif$ such that
    \[
        \advantage{\ctxtunif}{\OKEM}[(\adv)]
        \leq 1/\epsilon^\firstkeygensuccess_{\KEM,E}
        \cdot 1/\epsilon^\firstencapssuccess_{\KEM,E}
        \cdot \advantage{\sprcpa}{\KEM,\simulunif}[(\bdv)]
    \]
\end{lemma}
\begin{proof}
    To prove the reduction, we define three games, each representing a slightly modified version of the $\ctxtunif$ game:
    \begin{itemize}
        \item Let $G_0$ be the $b=0$ version of $\ctxtunif$, i.e. an adversary such as $\adv$ would receive $\pk, \hat c$ where 
        $(\pk,\sk) \getsr \OKEM.\kgen(), \hat c \getsr \bin^\obfctxtlen$.
        \item Let $G_1$ be the $b=1$ version of $\ctxtunif$, i.e. an adversary would receive $\pk, \hat c$ where\\
        $(\pk,\sk) \getsr \OKEM.\kgen(), (\hat c, K) \getsr \OKEM.\encaps(\pk)$.
        \item Let $G'$ be a mix of $G_0$ and $G_1$ employing the simulator instead of $\KEM.\encaps$, i.e. an adversary such as $\adv$ would receive $\pk, \hat c$ where\\
        $(\pk,\sk) \getsr \OKEM.\kgen(), \pcdo\ c \getsr \simulunif, \hat c \gets \encodectxt(c)\ \pcwhile \hat c = \bot$.
    \end{itemize}

    We analyze the probability of the $\ctxtunif$ adversary $\adv$ outputting the bit $1$ in each of these games, which we denote by $p_0, p_1$ and $p'$ respectively.
    We now have
    \[
        \advantage{\ctxtunif}{\OKEM}[(\adv)] = \left| p_0 - p_1 \right|
    \]

    Due to the rejection-sampling in $G'$, the sampling of $c$ in $G'$ is statistically indistinguishable from directly sampling $c \in \mathbf{C}'$. As $\encodectxt$ is bijective, it will output a uniformly random bit string upon input of a uniformly random element from $\mathbf{C}'$. This makes $G'$ and $G_0$ statistically indistinguishable, i.e.
    \[ p' = p_0. \]

    To bound the difference between $p_1$ and $p'$, we reduce to the $\sprcpa$ security of $\KEM$:
    Let $b$ be the bit from the $\sprcpa$ game. Let $\bdv$ be a reduction which, given $\pk, c, K$, computes $b \gets \filterpk(\pk)$, $\hat c \gets \encodectxt(c)$, aborts if $b=0$ or $\hat c = \bot$, and otherwise outputs $\adv(\pk, \hat c)$.
    Assuming $\bdv$ does not abort, this perfectly simulates either game $G_1$ (if $b=1$) or game $G'$ (if $b=0$) towards $\adv$. Recall that $\simulunif$ samples uniformly at random from $\mathbf{C}$.
    
    Let $\mathsf{ABRT}$ denote the event that $\bdv$ aborts, and let $b'$ be the output of $\adv$. Using advantage rewriting we can then express the advantage of $\bdv$ in the $\sprcpa$ game as follows:
    \begin{align*}
        \advantage{\sprcpa}{\KEM, \simulunif}[(\bdv)]
        &=
        2 \cdot \left|
            \Pr\left[b'=b \land \neg \mathsf{ABRT} \right] - \frac{1}{2}
        \right| \\
        &=
        \left|
            \Pr\left[b'=1 \land \neg \mathsf{ABRT} \mid b=1 \right] - \Pr\left[b'=1 \land \neg \mathsf{ABRT} \mid b=0 \right]
        \right| \\
        &=
        \Pr\left[\neg \mathsf{ABRT}\right] \cdot
        \left|
            \Pr\left[b'=1 \mid b=1 \land \neg \mathsf{ABRT} \right]
        \right.\\   
        &\qquad\qquad\qquad\  - \left.
            \Pr\left[b'=1 \mid b=0 \land \neg \mathsf{ABRT} \right]
        \right| \\
        &=
        \Pr\left[\neg \mathsf{ABRT}\right] \cdot
        \left|
            p_1 - p'
        \right|
    \end{align*}

    As the probability of aborting relates directly to first-keygen and first-encaps success probabilities, we conclude that
    \[
    \left| p_1 - p' \right| =
    1/\epsilon^\firstkeygensuccess_{\KEM,E}
    \cdot 1/\epsilon^\firstencapssuccess_{\KEM,E}
    \cdot \advantage{\sprcpa}{\KEM, \simulunif}[(\bdv)].
    \]
    
    Combining these results with the triangle inequality, we obtain the claimed bound.
\end{proof}

\subsection{Obfuscated Key Exchange}

In \cite{CCS:GunSteVei24}, the \pqobfs{} protocol is constructed for secure authenticated key exchange and achieves obfuscation. This protocol allows for the initial setup of a shared key and achieves various other desirable security properties needed for a FEP:
\pqobfs{} builds upon \obfsfour{} \cite{obfs4}, an established ``pluggable transport'' protocol for the Tor Project, and \pqobfs{} inherits features such as disguising flow signatures, probing resistance and confidentiality even if the node public key and node ID are known.
However, \pqobfs{} gives even stronger guarantees than \obfsfour{} when node public key and node ID are known, and can be instantiated with many OKEMs and not just the \textsf{Elligator2} construction~\cite{CCS:BHKL13}. This allows for the use of quantum-safe KEMs such as those standardized through the NIST Post-Quantum Cryptography Standardization \cite{nist-standardization} which was not possible in \obfsfour{} without major changes.

Furthermore, \cite{CCS:GunSteVei24} provides a formal security model and proofs for both \obfsfour{} and \pqobfs{} demonstrating the various guarantees.

To define the security of their obfuscated key exchange protocol, \cite{CCS:GunSteVei24} introduces the security notion of strong Obfuscated Key Exchange security ($\sObfKE$), capturing key indistinguishability, explicit server authentication, strong obfuscation, and probing resistance.
Strong obfuscation describes the protocol's ability to be unidentifiable (compared to a simulated protocol run) by an adversary, so long as the server's private key is not revealed. This is in contrast to the so-called regular obfuscation, achieved e.g.~by \obfsfour{}, which also requires the server's public key to remain secret.
Probing resistance, in turn, means that servers should not respond to messages from adversaries unless the server's public key was revealed \cite[Section~4.1]{CCS:GunSteVei24}.

In their follow-up work, \cite{EPRINT:GRSV25} constructed the \drivel{} protocol for the same general purpose. Compared to \pqobfs{}, \drivel{} only has one conceptual difference, in that it allows one of the two involved schemes to be a KEM instead of an OKEM.
This is accomplished by the derivation of symmetric encryption keys from the semi-secret server information combined with a shared secret generated from an OKEM encapsulation against the server's public key (which was distributed out-of-band).
The symmetric encryption keys can then be used to encrypt the KEM public key in the first message, and the KEM ciphertext in the server's reply.

This change has several advantages:
\begin{itemize}
    \item The use of a KEM for the client allows for shorter messages and more efficient computation. The overhead of symmetric encryption is small.
    \item This completely removes the need for obfuscated OKEM public keys: the server's public key is distributed out-of-band anyway, and the client's public key is now encrypted.
    \item If appropriate OKEM and KEM are used then \drivel{} can offer hybrid obfuscation, i.e.~\drivel{} can combine traditional and post-quantum schemes to hedge against security failures in either component.
\end{itemize}

The following theorem is a summary of \cite[Theorem~5]{EPRINT:GRSV25} and gives requirements on the functions used to instantiate \drivel{} in order to achieve $\sObfKE$ security. The instantiation requires a KEM, an OKEM, two functions $F_1, F_2$ used for authenticity as well as key derivation, and a symmetric encryption scheme.

When we construct OKEMs in later sections of this work, we will show that they achieve the required security notions listed below.

\begin{theorem}[Obfuscated Key Exchange Security]
\label{thm:drivel-security}
    Let \drivel{} be defined as in \cite[Fig.~6]{EPRINT:GRSV25}. Let $\OKEM$ be an OKEM. Let $\KEM$ be a KEM.
    
    The $\sObfKE$ security of \drivel{} is shown to reduce to the following properties of $\OKEM$:
    \begin{itemize}
        \item Low chance of public key collisions across all bridges, as measured by $\pkcoll{\OKEM}$
        \item KEM Correctness $\delta_\OKEM$
        \item $\indcca$ security of $\OKEM$
        \item $\sprcca$ security of $\OKEM$ (with respect to $\simulbin$)
    \end{itemize}

    In addition, the following properties of $\KEM$ are required:
    \begin{itemize}
        \item Low chance of public key collisions across all sessions, as measured by $\pkcoll{\KEM}$
        \item KEM Correctness $\delta_\KEM$
        \item $\indonecca$ security of $\KEM$
    \end{itemize}

    And, finally, the $\sObfKE$ security also depends on the $\prf$ security of $F_1$ and $F_2$, the $\swapprf$ security of $F_2$, and the symmetric encryption scheme's one-time ciphertext indistinguishability.
\end{theorem}

We instantiate $F_1,F_2$ and the symmetric encryption scheme for the \drivel{} protocol as described later in \cref{sec:drivel-instance}.

\begin{corollary}[Encapsulate-then-Encode Obfuscated Key Exchange Security]
\label{cor:drivel-security-filter-encode}
    It follows from \cref{thm:drivel-security,thm:filter-encode-security} that, for an OKEM constructed as $\OKEM = \feo[\KEM',E]$, the OKEM properties required for \drivel{} further reduce to the following:
    \begin{itemize}
        \item Low chance of public key collisions across all bridges, as measured by $\pkcoll{\KEM'}$
        \item KEM Correctness $\delta_\KEM'$
        \item $\indcca$ security of $\KEM'$
        \item $\sprcca$ security of $\KEM'$
        \item High first-keygen success probability $\epsilon^\firstkeygensuccess_{\KEM',E}$
        \item High first-encaps success probability $\epsilon^\firstencapssuccess_{\KEM',E}$
        \item $\ctxtunif$ security of $\feo[\KEM',E]$
    \end{itemize}
\end{corollary}

\subsection{The Drivel Protocol}

In this thesis, we implement and evaluate the \drivel{} protocol, which targets $\sObfKE$ security.
To make this document more self-contained, we reproduce the central definition of \drivel{} from \cite[Figure~6]{EPRINT:GRSV25} below.
For further information not reproduced in \cref{fig:drivel}, we refer readers to the original paper.

\begin{figure}
    % \begin{minipage}[t]{\textwidth}
	\hfill
	\pseudocode[codesize=\footnotesize,jot=-1mm]{%
		\textbf{Server key generation/setup} \\
		\nodeid \getsr \bin^{\nodeidlen} \\
		(\sk_S, \pk_S, \_) \getsr \OKEM.\kgen() \\
		\pstate.\macregister \gets \emptyset \\
		\text{return } ((\sk_S, \nodeid), (\pk_S, \nodeid), \pstate)
	}
	\hspace*{0.25cm}
% \end{minipage}

\vspace{0.5em}

% \hspace*{-0.2in}
\scalebox{0.9}{%
\begin{tikzpicture}
	% Set the X coordinates of the client, server, and arrows
	\edef\ClientX{0}
	%% full page
	\edef\ArrowLeft{3}
	\edef\ArrowRight{13}
	\edef\ServerX{16.5}
	\edef\ServerLeftTextwidth{10.1cm} % width of server-side text, when left-aligned
	%% CCS one column
	% \edef\ArrowLeft{1}
	% \edef\ArrowRight{9}
	% \edef\ServerX{10}
	% \edef\ServerLeftTextwidth{8.25cm} % width of server-side text, when left-aligned
	% Set the starting Y coordinate
	\edef\Y{0}

	% Draw header boxes
	\node [rectangle,draw,inner sep=5pt,right] at (\ClientX,\Y) {\textbf{Client}};
	\node [rectangle,draw,inner sep=5pt,left] at (\ServerX,\Y) {\textbf{Server}};
	
	% \NextLine[0.4]
	% \ClientAction[gray,font=\small]{\hspace{1.25cm}knows $(\pk_S, \nodeid)$}
	% \ServerAction[gray,font=\small]{knows $(\sk_S, \nodeid)$\hspace{1.25cm}\null}
	% \NextLine[1.1]
	
	\NextLine[0.4]
	\ClientAction[gray,font=\small]{\hspace{1.5cm}knows $(\pk_S, \nodeid)$}
	\ServerAction[gray,font=\small]{knows $(\sk_S, \nodeid)$\hspace{1.5cm}\null}
	\NextLine[1.1]
	
	\ClientAction{$(\sk_e, \pk_e) \getsr \highlightbox{\KEM}.\kgen()$} 
	\NextLine
	\ClientAction{$P_C \getsr P_Cdist$}
	\NextLine
	% \ClientAction{\old{$(c_S, K_S) \getsr \KEM.\encaps(\pk_S)$}}
	% \NextLine
	% \ClientAction{\old{$cobf_S \gets \ObfEncode(c_S)$}}
	% \NextLine
	\ClientAction{$(c_S, K_S) \getsr \OKEM.\encaps(\pk_S)$}
	\NextLine
	% \ClientAction{$cobf_S \getsr \OKEM.\ObfCTEnc(c_S)$}
	% \NextLine
	% \ClientAction{\old{$ES \highlightbox{$\conc EK_1 \conc EK_2$} \gets \funComb(\nodeid, K_S)$}}
	% \NextLine
	\ClientAction{$ES \gets \funComb(\nodeid, K_S)$}
	\NextLine
	\ClientAction{\highlightbox{$EK_1 \gets \funPRF(ES, \textlit{:enckey1}, \keylen_1)$}}
	\NextLine
	\ClientAction{\highlightbox{$EK_2 \gets \funPRF(ES, \textlit{:enckey2}, \keylen_2)$}}
	\NextLine
	\ClientAction{\highlightbox{$epk_e \gets \funEnc(EK_1, \pk_e)$}}
	\NextLine
	\ClientAction{$M_C \gets \funPRF(ES, epk_e \conc c_S \conc \textlit{:mc})$}
	%\mar{why do we have this? the previous paper even admits M_C is unnecessary}\mar{because M_S is necessary when the server has content after its response, and this is symmetric}}
	\NextLine
	\ClientAction{$\mathsf{MAC}_C \gets \funPRF(ES, epk_e \conc c_S \conc P_C \conc M_C \conc \textlit{:mac\_c})$}
	
	%%%%%%%%%%%%%%%%%%%%%%%%%%%%%%%%%%%%%%%%%%%%%%%%%%%%%%%%%%%%%%%%%%%%%%%%%%%%%%%%%%%%%%%%%%%%%%%%%%%%%%%
	\NextLine[1.5]
	\ClientToServer{$\mathsf{msg}_C = epk_e \conc c_S \conc P_C \conc M_C \conc \mathsf{MAC}_C$}{}
	\NextLine[1]
	%%%%%%%%%%%%%%%%%%%%%%%%%%%%%%%%%%%%%%%%%%%%%%%%%%%%%%%%%%%%%%%%%%%%%%%%%%%%%%%%%%%%%%%%%%%%%%%%%%%%%%%
	
	\ServerActionLeft{$epk_e \gets \mathsf{msg}_C[1..\obfpklen]$ ; $c_S \gets \mathsf{msg}_C[\obfpklen+1..\obfpklen+\obfctxtlen]$}
	\NextLine
	% \ServerActionLeft{\old{$c_S \gets \ObfDecode(cobf_S)$}}
	% \NextLine
	% \ServerActionLeft{\old{$K_S \gets \KEM.\decaps(\sk_S, c_S)$}}
	% \NextLine
	% \ServerActionLeft{$c_S \gets \OKEM.\ObfCTDec(cobf_S)$}
	% \NextLine
	\ServerActionLeft{$K_S \gets \OKEM.\decaps(\sk_S, c_S)$}
	\NextLine
	% \ServerActionLeft{\old{$ES \highlightbox{$\conc EK_1 \conc EK_2$} \gets \funComb(\nodeid, K_S)$}}
	% \NextLine
	\ServerActionLeft{$ES \gets \funComb(\nodeid, K_S)$}
	\NextLine
	\ServerActionLeft{\highlightbox{$EK_1 \gets \funPRF(ES, \textlit{:enckey1}, \keylen_1)$}}
	\NextLine
	\ServerActionLeft{\highlightbox{$EK_2 \gets \funPRF(ES, \textlit{:enckey2}, \keylen_2)$}}
	\NextLine
	\ServerActionLeft{$M_C \gets \funPRF(ES, epk_e \conc c_S \conc \textlit{:mc})$}
	\NextLine[1.2]
	% \ServerActionLeft{$M_C \gets \HMAC(\cpk^\ltssk, \ltspk \conc \nodeid \conc \ellcpk)$}
	% \NextLine
	\ServerActionLeft{parse $(epk_e \conc c_S \conc P_C \conc M_C \conc \mathsf{MAC}_C) \gets \mathsf{msg}_C$ using $M_C$; else $\texttt{break}$}
	\NextLine
	\ServerActionLeft{if $\funPRF(ES, epk_e \conc c_S \conc P_C \conc M_C \conc \textlit{:mac\_c}) \neq \mathsf{MAC}_C$: $\texttt{break}$}
	\NextLine[1.2]
	\ServerActionLeft{if $\mathsf{MAC}_C \in \pstate.\macregister$: $\texttt{break}$}
	\NextLine
	\ServerActionLeft{$\pstate.\macregister \gets \pstate.\macregister \cup \{\mathsf{MAC}_C\}$}
	\NextLine
	\ServerActionLeft{\highlightbox{$\pk_e \gets \funDec(EK_1, epk_e)$}}
	\NextLine
	% \ServerActionLeft{\old{$(c_e, K_e) \getsr \KEM.\encaps(\pk_e)$}}
	% \NextLine
	% \ServerActionLeft{\old{$cobf_e \gets \ObfEncode(c_e)$}}
	% \NextLine
	\ServerActionLeft{$(c_e, K_e) \getsr \highlightbox{\KEM}.\encaps(\pk_e)$}
	\NextLine
	\ServerActionLeft{\highlightbox{$ect_e \gets \funEnc(EK_2, c_e)$}}
	\NextLine
% 	\ServerActionLeft{$\text{\codecomment{ntor handshake component}}$}
% 	\NextLine
	\ServerActionLeft{$\mathsf{protoID} \gets \highlightbox{\textlit{drivel}}$}
	\NextLine
	\ServerActionLeft{$ES' \gets \funPRF(ES, \textlit{:derive\_key})$ ; $FS \gets \funComb(ES', K_e)$}
	\NextLine
% 	\ServerActionLeft{$FS \gets \funComb(ES', K_e)$}
% 	\NextLine
	\ServerActionLeft{$\mathsf{context} \gets \pk_S \conc c_S \conc \pk_e \conc c_e \conc \mathsf{protoID}$}
	\NextLine
	\ServerActionLeft{$\mathsf{KEY\_SEED} \gets \funPRF(FS, \mathsf{context} \conc \textlit{:key\_extract})$}
	\NextLine
	\ServerActionLeft{$\mathsf{auth} \gets \funPRF(FS, \mathsf{context} \conc \textlit{:server\_mac})$}
	\NextLine
% 	\ServerActionLeft{$\text{\codecomment{end ntor handshake component}}$}
% 	\NextLine
	\ServerActionLeft{$P_S \getsr P_Sdist$} % \text{random bytes of length }[0,8096]$}
	\NextLine
	\ServerActionLeft{$M_S \gets \funPRF(ES, ect_e \conc \textlit{:ms})$}
	\NextLine
	\ServerActionLeft{$\mathsf{MAC}_S \gets \funPRF(ES, ect_e \conc \mathsf{auth} \conc P_S \conc M_S \conc \textlit{:mac\_s})$}
	
	%%%%%%%%%%%%%%%%%%%%%%%%%%%%%%%%%%%%%%%%%%%%%%%%%%%%%%%%%%%%%%%%%%%%%%%%%%%%%%%%%%%%%%%%%%%%%%%%%%%%%%%
	\NextLine[1.5]
	\ServerToClient{$\mathsf{msg}_S = ect_e \conc \mathsf{auth} \conc P_S \conc M_S \conc \mathsf{MAC}_S$}{}
	\NextLine[1]
	%%%%%%%%%%%%%%%%%%%%%%%%%%%%%%%%%%%%%%%%%%%%%%%%%%%%%%%%%%%%%%%%%%%%%%%%%%%%%%%%%%%%%%%%%%%%%%%%%%%%%%%
	
	\ClientAction{$ect_e \gets \mathsf{msg}_S[1..\obfctxtlen]$}
	\NextLine
	\ClientAction{$M_S \gets \funPRF(ES, ect_e \conc \textlit{:ms})$}
	\NextLine
	\ClientAction{parse $(ect_e \conc \mathsf{auth} \conc P_S \conc M_S \conc \mathsf{MAC}_S) \gets \mathsf{msg}_S$ using $M_S$; else $\texttt{break}$}
	\NextLine
	\ClientAction{if $\funPRF(ES, ect_e \conc \mathsf{auth} \conc P_S \conc M_S \conc \textlit{:mac\_s}) \neq \mathsf{MAC}_S$: $\texttt{break}$}
	\NextLine
	% \ClientAction{\old{$c_e \gets \ObfDecode(cobf_e)$}}
	% \NextLine
	% \ClientAction{\old{$K_e \gets \KEM.\decaps(\sk_e, c_e)$}}
	% \NextLine
	\ClientAction{\highlightbox{$c_e \gets \funDec(EK_2, ect_e)$}}
	\NextLine
	\ClientAction{$K_e \gets \highlightbox{\KEM}.\decaps(\sk_e, c_e)$}
	\NextLine
% 	\ClientAction{$\text{\codecomment{ntor handshake component}}$}
% 	\NextLine
	\ClientAction{$\mathsf{protoID} \gets \highlightbox{\textlit{\drivel{}}}$}
	\NextLine
	\ClientAction{$ES' \gets \funPRF(ES, \textlit{:derive\_key})$ ; $FS \gets \funComb(ES', K_e)$}
	\NextLine
% 	\ClientAction{$FS \gets \funComb(ES', K_e)$}
% 	\NextLine
	\ClientAction{$\mathsf{context} \gets \pk_S \conc c_S \conc \pk_e \conc c_e \conc \mathsf{protoID}$}
	\NextLine
	\ClientAction{$\mathsf{KEY\_SEED} \gets \funPRF(FS, \mathsf{context} \| \textlit{:key\_extract}) $}
	\NextLine
	\ClientAction{if $\funPRF(FS, \mathsf{context} \conc \textlit{:server\_mac}) \neq \mathsf{auth}$: $\texttt{break}$}
\end{tikzpicture}
}

    \caption[
        The \drivel{} obfuscated key exchange protocol.
    ]{
        The \drivel{} obfuscated key exchange protocol.
        $\OKEM$ is an OKEM satisfying $\indcca, \sprcca$ security, and ciphertext uniformity.
        $KEM$ is an $\indonecca$-secure KEM.
        $SE$ is a length-preserving, one-time ciphertext indistinguishable, symmetric encryption scheme.
        $F_1$ is a PRF and $F_2$ is a dual PRF.
        This Figure is a reproduction of \cite[Figure~6]{EPRINT:GRSV25} but without highlights and slight changes to the KEM and OKEM syntax.
    }
    \label{fig:drivel}
\end{figure}