\chapter{Preliminaries}\label{ch:preliminaries}

\paragraph{Notation}
In pseudocode, we use the notation $x \gets A(y)$ to express an assignment to a variable $x$ using the output of the deterministic computation $A(y)$. When the computation is probabilistic, we write $x \getsr A(y)$.
Uniformly random sampling of $x$ from a set $S$ is denoted $x \getsr S$.

TODO: Define basic constructs like PRF, etc if needed.

\section{Review of Prior Work} \label{sec:review-gsv24}

Günther et al. first \cite{CCS:GunSteVei24} described a framework for proveable security of obfuscated key exchange, extending efforts by Fenske and Johnson \cite{CCS:FenJoh24}, whose work focused on the data transfer phase of fully encrypted protocols.

The work by Günther et al. was iterated upon in \cite{EPRINT:GRSV25} where the notation was simplified, and the proposed protocol from \cite{CCS:GunSteVei24} was slightly modified to yield \drivel, the obfuscated key exchange protocol we will implement in this thesis.

For convenience, we reproduce the definitions from \cite{CCS:GunSteVei24,EPRINT:GRSV25} that are relevant for this thesis here. This reproduction contains equivalent definitions, but we omit the concept of obfuscated public keys and focus only on ciphertext obfuscation, as public keys are only transmitted after encryption or out-of-band in \drivel.

\subsection{Key Encapsulation Mechanisms}

\begin{definition}[Key encapsulation mechanism]
    \label{def:kem}
    A key encapsulation mechanism $\KEM = (\kgen, \encaps, \decaps)$ consists of three algorithms:
    \begin{itemize}
        \item $\ul{\kgen}() \tor (\pk, \sk)$
        is a probabilistic \emph{key generation} algorithm that generates a public key~$\pk$ and corresponding secret key~$\sk$.
        \item $\ul{\encaps}(\pk) \tor (c, K)$
        is a probabilistic \emph{encapsulation} algorithm that takes as input a KEM public key~$\pk$, and outputs a ciphertext~$c$ and shared secret~$K$.
        \item $\ul{\decaps}(\sk, c) \to K$
        is a deterministic \emph{decapsulation} algorithm that takes as input a secret key~$\sk$ and ciphertext~$c$, and outputs a shared secret~$K$.
    \end{itemize}
\end{definition}

\begin{definition}[KEM correctness]
\label{def:kem-corr}
We say that a KEM~$\KEM=(\kgen, \encaps, \decaps)$ is \emph{$\delta_\KEM$-correct} if
\[
    \Pr\left[
        \decaps(\sk,c) \neq K
    ~\middle|
        \begin{array}{c}
        (\pk,\sk) \getsr \kgen(), \\
        (c, K) \getsr \encaps(\pk)
        \end{array}
    \right] \leq \delta_\KEM.
\]
\end{definition}

To define what it means for a KEM to be secure, we require two security notions. One focuses on indistinguishability of ciphertexts ($\indcca$), i.e. given a public key and ciphertext, it should be hard to differentiate the encapsulated shared secret from uniformly random elements of the key space.
This captures the goal that an adversary (even given the public key $\pk$) does not learn any information about the shared secret encapsulated in the challenge ciphertext $c^*$, even when allowed to obtain decapsulations of other ciphertexts.
This notion is central to the main premise of KEMs, namely to initialize a secure communication channel given only one party's public key by way of setting up a shared secret.

A second notion is strong pseudorandomness ($\sprcca$), i.e. given a public key, it should be hard to differentiate a ciphertext-key pair obtained from encapsulation for that public key from a pair consisting of a randomly sampled ciphertext and a uniformly randomly sampled key, even when allowed to obtain decapsulations of other ciphertexts.
This captures the goal that ciphertexts must "look randomly chosen" among a set of possible ciphertexts and should look unrelated to any given public key - even given the corresponding shared secret. A KEM can be $\sprcca$ secure with respect to a subset of the full ciphertext space, and the sampling of ciphertexts does not have to be uniformly at random. This is expressed throught the use of a simulator $\mathcal S$.
This notion is first introduced \cite{EC:Xagawa22} in this form by Xagawa, and it is satisfied by many KEMs that are being considered for use in quantum-safe communication \cite{EC:Xagawa22}. We use $\sprcca$ security as a stepping stone to argue about the distribution of ciphertext bits.

\begin{definition} \label{def:kem-security}
    We define two security notions $\indcca$ and $\sprcca$ using the games shown in \cref{fig:kem-security} and we define the respective advantages of an adversary~$\adv$ against the $\mathsf{goal} \in \{\indcca, \sprcca\}$ security of a KEM~$\KEM$ as
\[
    \advantage{\mathsf{goal}}{\KEM}[(\adv)] := 2 \cdot \left| \Pr \left[ G^{\mathsf{goal}}_{\KEM}(\adv) \Rightarrow 1 \right] - \frac{1}{2} \right|.
\]
\end{definition}

\begin{figure}
    \begin{pchstack}[boxed, center, space=0.5em]

\begin{pcvstack}[space=0.05em]
\procedure[linenumbering]{$\textbf{GAME } \indcca(\adv)$}{
    b \getsr \bin \\
    (\pk, \sk) \getsr \kgen() \\
    (c^*, K_0) \getsr \encaps(\pk) \\
    K_1 \getsr \mathcal K \\
    b' \getsr \adv^{\decapsoracle}(\pk, c^*, K_b) \\
    \pcreturn \llbracket b = b'\rrbracket
}

\procedure[lnstart=6,linenumbering]{$\decapsoracle(c \neq c^*)$}{
    K \gets \decaps(c) \\
    \pcreturn K
}

\end{pcvstack}

\begin{pcvstack}[space=0.05em]
\procedure[lnstart=8,linenumbering]{$\textbf{GAME } \sprcca(\adv)$}{
    b \getsr \bin \\
    (\pk, \sk) \getsr \kgen() \\
    (c_0, K_0) \getsr \encaps(\pk) \\
    c_1 \getsr \mathcal S \\
    K_1 \getsr \mathcal K \\
    c^* \gets c_b \\
    b' \gets \adv^{\decapsoracle}(\pk, c_b, K_b) \\
    \pcreturn \llbracket b = b'\rrbracket
}
\end{pcvstack}

\end{pchstack}

    \caption[%
        Security games for $\indcca$ and $\sprcca$ security of a KEM or obfuscated KEM.
    ]{%
        Security games for $\indcca$ and $\sprcca$ security of a KEM~$\KEM = (\kgen, \encaps, \decaps)$ with key space~$\mathcal K$. $\sprcca$ security is defined with respect to a simulator $\mathcal S$. Oracle queries violating the stated condition on arguments are not allowed.
    }
    \label{fig:kem-security}
\end{figure}

\begin{definition}[KEM public key collision probability]
    \label{def:pk-collisions}
    Let $\KEM$ be a KEM.
    We define the \emph{public key collision probability} of $\KEM$ for $n \in \mathbb{N}$ public keys as
    \[
        \pkcoll{\KEM}(n) := \Pr\left[
            \begin{array}{c}
                \pk_i = \pk_j \\
                \land~ i \neq j
            \end{array}
            \middle|
            \begin{array}{c}
                (\pk_i,\sk_i) \getsr \KEM.\kgen() \\
                \text{ for } i \in [1,n]
            \end{array}
            \right].
    \]
\end{definition}

\subsection{Obfuscated KEMs}

\begin{definition}[Obfuscated KEM]
    \label{def:okem}
    An \emph{obfuscated key encapsulation mechanism (OKEM)} $\OKEM = (\kgen, \encaps, \decaps, \decode)$ with \emph{obfuscated ciphertext length}~$\obfctxtlen \in \mathbb{N}$ consists of the following algorithms:
    \begin{itemize}
        \item $\ul{\kgen}() \tor (\pk,\sk)$ is a probabilistic \emph{key generation} algorithm that generates a public key~$\pk$ and corresponding secret key~$\sk$.

        \item $\ul{\encaps}(\pk) \tor (\hat c, K)$ is a probabilistic \emph{encapsulation} algorithm that takes as input an OKEM public key~$\pk$ and outputs an (obfuscated) ciphertext~$\hat c \in \bin^\obfctxtlen$ and shared secret~$K$.

        \item $\ul{\decaps}(\sk,\hat c) \to K$ is a deterministic \emph{decapsulation} algorithm that takes as input a secret key~$\sk$ and (obfuscated) ciphertext~$\hat c$, and outputs a shared secret~$K$.
    \end{itemize}
    Note that the tuple of algorithms $(\kgen, \encaps, \decaps)$ can also be viewed as a KEM as in \cref{def:kem}, but we restrict the ciphertext space to bitstrings.

    As in \cref{def:kem}, we also demand KEM correctness for OKEMs.
\end{definition}

The $\indcca$ and $\sprcca$ security of an OKEM, its correctness, and the public key collision probability are defined exactly as for the underlying KEM (see \cref{fig:kem-security}, \cref{def:kem-corr,def:pk-collisions}).

In addition to the above properties, which an OKEM should also fulfill, an obfuscated KEM should fulfill stronger randomness requirements. In particular, its public keys and ciphertexts should not be distinguishable from random bit strings. This is in contrast to regular KEMs which may have recognizable structures in $\pk, c$ even if they are $\sprcca$-secure.

\begin{definition}[Ciphertext uniformity]\label{def:ctxt-uniformity}
    Let $\OKEM$ be an OKEM.
    We measure the \emph{ciphertext uniformity} of the obfuscated ciphertext of length~$\obfctxtlen$ generated by $\OKEM.\encaps$ against an adversary~$\adv$ as
    \[
        \advantage{\ctxtunif}{\OKEM}[(\adv)] := 
        2 \cdot \left|
        \Pr\left[
            \adv(\pk, \hat c_b) = b
        ~\middle|
            \begin{array}{c}
                b \getsr \bin, \hat c_1 \getsr \bin^{\obfctxtlen}, \\
                (\pk,\sk) \getsr \OKEM.\kgen(),\\
                (\hat c_0, K_0) \getsr \OKEM.\encaps(\pk)
            \end{array}
        \right]
        - \frac{1}{2}
        \right|
    \]
    
    For an unbounded adversary~$\adv$, we call the advantage~$\advantage{\ctxtunif}{\OKEM}[(\adv)] $ \emph{statistical}.
\end{definition}

\subsection{OKEM construction}

To construct an OKEM, a regular KEM can be augmented with encoding functions like \textsf{Elligator2}~\cite{CCS:BHKL13} or \textsf{Kemeleon}~\cite[Sec. 2.4]{CCS:GunSteVei24}. We define the following construct to bundle such functions.

\begin{definition}[KEM Encoding]
\label{def:kem-encoding}
    Let $\KEM = (\kgen, \encaps, \decaps)$ be a KEM.
    Let $\obfctxtlen \in \mathbb N$.
    We then define a \emph{KEM encoding} for $\KEM$ to consist of the following algorithms operating on the appropriate input/output spaces of $\KEM$:
    \begin{itemize}
        \item $\ul{\encodectxt}(c) \to \hat c$
        is a (possibly randomized) \emph{ciphertext encoding} algorithm that on input of a ciphertext~$c$ from the ciphertext space of $\KEM$ outputs an obfuscated ciphertext~$\hat c \in \bin^{\obfctxtlen}$ or an error $\bot$.
        \item $\ul{\decodectxt}(\hat c) \to c$
        is a deterministic \emph{ciphertext decoding} algorithm that on input of an obfuscated ciphertext~$\hat c \in \bin^{\obfctxtlen}$ outputs a ciphertext~$c$ from the ciphertext space of $\KEM$.
    \end{itemize}

    We demand that these encoding/decoding algorithms are perfectly correct, i.e.:
    \[
        \Pr\left[
            \decodectxt(\hat c) = c
        ~\middle|
            \begin{array}{c}
                (\pk, \sk) \getsr \kgen(),\\
                (c, K) \getsr \encaps(\pk),\\
                \hat c \getsr \encodectxt(c),\\
                \hat c \neq \bot
            \end{array}
        \right] = 1.
    \]
\end{definition}

\begin{definition}[First-encaps Success Probability]
\label{def:first-encaps-success}
    Let $\KEM = (\kgen, \encaps, \decaps)$ be a KEM.
    Let $\encodectxt, \decodectxt$ be a KEM encoding for $\KEM$.
    We then define the \emph{first-encaps success probability} as the probability that \cref{ln:encaps-encode} was executed only once in a given execution of $\encaps'$, i.e.
    \[
        \epsilon^\firstencapssuccess_\OKEM :=
        \Pr\left[
            \encodectxt(c) \neq \bot
        ~\middle|
            \begin{array}{c}
                (\pk, \sk) \getsr \kgen(), \\
                c \getsr \encaps(\pk)
            \end{array}
        \right]
    \]
\end{definition}

\begin{definition}[Encapsulate-then-Encode Transformation]
\label{def:encaps-then-encode}
    Let $\KEM = (\kgen, \encaps, \decaps)$ be a KEM.
    Let $\encodectxt, \decodectxt$ be a KEM encoding for $\KEM$.
    We then define the corresponding \emph{encapsulate-then-encode} obfuscated KEM $\ete[\KEM, \encodectxt, \decodectxt] = (\KEM.\kgen, \encaps', \decaps')$ with the following new algorithms:

    \begin{pchstack}[boxed, center, space=0.5em]

    \procedure[linenumbering]{$\kgen'()$}{
        \pcdo \\
        \t  (\pk, \sk) \getsr \KEM.\kgen() \\
        \label{ln:kgen-encode}
        \t  \pkobf \getsr \encode(\pk) \\
        \pcwhile \pkobf = \bot \\
        \pcreturn (\pkobf, \sk)
    }
    
    \procedure[lnstart=5,linenumbering]{$\encaps'(\pkobf)$}{
        \pk \gets \decode(\pkobf) \\
        \pcdo \\
        \t  (c, K) \getsr \KEM.\encaps(\pk) \\
        \label{ln:encaps-encode}
        \t  \hat c \getsr \encodectxt(c) \\
        \pcwhile \hat c = \bot \\
        \pcreturn (\hat c, K)
    }
    
    \procedure[lnstart=11,linenumbering]{$\decaps'(\sk, \hat c)$}{
        c \gets \decodectxt(\hat c) \\
        \pcreturn \KEM.\decaps(c)
    }

\end{pchstack}
\end{definition}

Note that in this construction, the first-encaps success probability from \cref{def:first-encaps-success} is precisely the probability that \cref{ln:encaps-encode} was executed only once in a given execution of $\encaps'$.

For an OKEM constructed according to \cref{def:encaps-then-encode} using appropriate encoding/decoding functions, it was shown in \cite[Theorems 2.12 and 2.13]{CCS:GunSteVei24} that the central security notions are preserved:

\begin{theorem}\label{thm:encaps-then-encode-security}
    Let $\OKEM$ be a encapsulate-then-encode obfuscated KEM based on a regular KEM $\KEM$ as per \cref{def:encaps-then-encode}.
    For any adversary $\adv$ against the $\indcca$ security of $\OKEM$, there exists an algorithm $\bdv$ such that
    \[
        \advantage{\indcca}{\OKEM}[(\adv)]
        \leq
        1/\epsilon^\firstencapssuccess_\OKEM
        \cdot \advantage{\indcca}{\KEM}[(\bdv)]
    \]

    Further, for any adversary $\adv$ against the $\sprcca$ security of $\OKEM$, there exist algorithms $\bdv$, $\cdv$ such that
    \begin{align*}
        \advantage{\sprcca}{\OKEM}[(\adv)]
        \leq\ 
        & 1/\epsilon^\firstencapssuccess_\OKEM
        \cdot \advantage{\sprcca}{\KEM, \mathcal S_\mathsf{unif}}[(\bdv)] \\
        &+ \advantage{\ctxtunif}{\OKEM}[(\cdv)]
    \end{align*}
    
    where $\mathcal S_\mathsf{unif}$ is a ``uniform-encapsulation'' simulator for KEM, i.e. a simulator that outputs uniformly random ciphertexts from the appropriate output space.
\end{theorem}

Finally, the ciphertext uniformity of the resulting OKEM may be analyzed directly on the output distribution of the $\encodectxt$ algorithm.

TODO: continue from here

\subsection{Obfuscated Key Exchange}

In \cite{CCS:GunSteVei24}, a protocol for secure, authenticated key exchange using obfuscation is constructed ($\pqobfs$). This protocol allows for the initial setup of a shared key and achieves various other desirable security properties needed for a FEP.
This new $\pqobfs$ protocol builds upon $\obfsfour$, which already disguises flow signatures, has features for probing resistance and still guarantees some level of security if the node public key and node ID are known.
However, $\pqobfs$ comes with a formal security proof, it gives stronger guarantees when node public key and node ID are known and may also employ OKEMs with other constructions besides \textsf{Elligator2}~\cite{CCS:BHKL13}, enabling quantum-safe constructions.

To define the security of their obfuscated key exchange protocol, \cite{CCS:GunSteVei24} introduces the security notion $\sObfKE$. The following shortened theorem gives requirements on the employed OKEM which allow $\pqobfs$ to achieve this security goal.

When we construct OKEMs in later sections, we will show that they achieve the required security notions listed below.

\begin{theorem}\label{thm:s-obfuscated-keyex-security}
    Let $\pqobfs$ be defined as in \cite[Fig. 7]{CCS:GunSteVei24}. Let $\OKEM$ be an OKEM.
    Let $n_s, n_r$ be the number of sessions and servers respectively, and let $q_C$ be a bound on the number of real-or-random key exchanges which are available to the adversary in any given game.
    
    The $\sObfKE$ security of $\pqobfs$ is shown to reduce to the following properties of $\OKEM$, captured with the according term reproduced to see scaling factors:
    \begin{itemize}
        \item Low chance of public key collisions: $2 \cdot \pkcoll{\OKEM}(n_s+n_r)$
        \item Correctness: $4n_s \cdot \delta_\OKEM$
        \item Indistinguishability over all sessions: $2 n_s n_r \cdot \advantage{\indcca}{\OKEM}[]$
        \item Indistinguishability in a randomly chosen session: $(n_s^2 + n_s n_r q_C) \cdot \advantage{\indonecca}{\OKEM}[]$
        \item Pseudorandomness: $(n_s + n_r q_C) \cdot (\advantage{\sprcca}{\OKEM}[] + \advantage{\pkunif}{\OKEM}[] + \advantage{\ctxtunif}{\OKEM}[])$
    \end{itemize}

    and, in addition, the $\sObfKE$ security also depends on the $\prf$ and $\swapprf$ security of two pseudorandom functions, both of which we instantiate with HMAC, whose required security for fixed-length keys is proven in \cite{C:BBGS23}.
\end{theorem}