\chapter{Preliminaries}\label{ch:preliminaries}

\section{\texorpdfstring{Review of Günther, Stebila, and Veitch \cite{CCS:GunSteVei24}}{Review of Günther, Stebila, and Veitch}} \label{sec:review-gsv24}

Günther et al. first described a framework for proveable security of obfuscated key exchange, extending efforts by Fenske and Johnson \cite{CCS:FenJoh24}, whose work focused on the data transfer phase of fully encrypted protocols.

For convenience, we reproduce the central definitions, that this thesis interfaces with, here.

\begin{definition}\label{def:first-keygen-success}
    2.3 first keygen success prob.
\end{definition}

\begin{definition}\label{def:pk-uniformity}
    2.4 public key uniformity
\end{definition}

TODO: Note that even though 2.3 and 2.4 are defined for obfuscated key generation only, the definitions are compatible with OKEM.

\begin{definition}\label{def:okem}
    2.5 OKEM
\end{definition}

\begin{definition}\label{def:okem-correctness}
    2.6 OKEM correctness
\end{definition}

\begin{definition}\label{def:pk-collisions}
    2.7 public key collisions
\end{definition}

\begin{definition}\label{def:keygen-encap-then-encode}
    2.8 Keygen/Encap-then-encode OKEM construction
\end{definition}

\begin{definition}\label{def:first-encap-success}
    2.9 First encap success prob.
\end{definition}

\begin{definition}\label{def:ctxt-uniformity}
    2.10 ciphertext uniformity
\end{definition}

\begin{definition}\label{def:ind-spr-cca}
    IND-CCA, SPR-CCA for a KEM (note that simulator doesn't matter so long as proof goes through)
\end{definition}

\begin{theorem}\label{thm:s-obfuscated-keyex-security}
    7.1 For full protocol security sObfKE; list all requirements
\end{theorem}