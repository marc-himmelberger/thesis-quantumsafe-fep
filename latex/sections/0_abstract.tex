\begin{abstract}
The Tor network enables anonymous communication by routing traffic through volunteer-operated relays, but access to the network is often restricted through censorship that targets identifiable protocol patterns.
In order to counteract such restrictions, protocols must resist fingerprinting, for example, by obfuscating their traffic to appear like random bytes on the wire.

In this thesis, we build on the Tor Project's lyrebird repository, which implements the obfuscation protocol \obfsfour{} \cite{obfs4}. We implement the obfuscated key exchange protocol \drivel{} \cite{EPRINT:GRSV25} and integrate it into lyrebird. The \drivel{} protocol extends the \obfsfour{} protocol with post-quantum security and stronger obfuscation.
We analyze the Classic-McEliece and HQC key encapsulation mechanisms, defining mappings from ciphertexts to random byte strings required for their deployment in \drivel{}. These schemes are then integrated with our implementation, in addition to the previously examined ML-KEM and FrodoKEM schemes.
Finally, we empirically evaluate our implementation using benchmarks and deployments under laboratory conditions and suggest concrete changes to \drivel{} to reduce identifiable traffic patterns in the handshake and increase performance.
\end{abstract}

\clearpage
\section*{Acknowledgement}
I would like to extend my thanks first to my supervisors Shannon Veitch, Dr.~Felix Günther and Prof.~Dr.~Kenny Paterson for guiding me towards this engaging topic and giving me the opportunity to apply my skills for both theoretical analyses as well as a practical implementation. Their ideas and feedback in every discussion served to improve my thesis tremendously and I am grateful for the chance to contribute to such a relevant topic.

In addition, I want to thank Dr.~Keita Xagawa for his time and insights, which helped me better understand prior research in the space of anonymity of quantum-safe KEMs.

Finally, I thank my friends, colleagues, family, my fiancée, and our cat --- for keeping me motivated, for listening to many overly-technical explanations of my thesis, and for supporting me in more ways than I can count throughout the entirety of my studies.