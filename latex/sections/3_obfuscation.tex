\chapter{Obfuscating NIST candidates}\label{ch:obfuscation}

In this section, we construct encodings for different post-quantum KEMs suitable for use in the pq-obfs protocol.

While such encodings have already been constructed for elliptic curve Diffie–Hellman \cite{CCS:BHKL13, tor-dev-udh, USENIX:WWGH11} as well as for ML-KEM \cite{CCS:GunSteVei24}, other KEMs are also being examined for future use.

TODO: Also cite 64, 62 from Obfuscated Key Exchange? Cite by name?

We take a selection of round 4 candidates of the NIST Post-Quantum Cryptography Standardization, as well as previous candidate schemes which we believe to be of interest.
For each scheme, we describe the generation of public keys and ciphertexts in the original scheme and examine their distributions. We then introduce encoding algorithms from the original public keys and ciphertexts to bitstrings and finally give concrete bounds for uniformity and encoding success rate.

We call all our encoding schemes TODO where TODO is the name of the corresponding KEM. \cref{tab:obfuscation-summary} summarizes the result of the later sections.

TODO: Show domain in table? Update references to convenience definitions earlier.

\begin{table}
    \centering
    \scriptsize\raggedright
    \begin{tabular}{@{} *{2}{L{0.125\textwidth-\tabcolsep}} | *{3}{L{0.25\textwidth-2\tabcolsep}} @{}}
        \textbf{KEM} & \textbf{Encoding} & \textbf{Obfuscation/Ciphertext uniformity} & \textbf{First-Keygen/Encap Success Probability} (\cref{def:first-keygen-success} and \ref{def:first-encap-success}) & \textbf{Output Size} (in bytes)\\ \hline
        Classic McEliece \cite{NISTPQC-R4:ClassicMcEliece22} & TODO &  &  & \\
         & - public keys & 0.01 & 0.2 & 0.3 \\
         & - ciphertexts & 0.01 & 0.2 & 0.3 \\
         &  &  &  & \\
    \end{tabular}
    \caption{Summary of KEMs, their corresponding encodings and the results of our analysis. The origins of analysis results are specified, and for output sizes, differences in bytes from original public key/ciphertext sizes are given. This table can be viewed as an extension of \cite[Table~2]{CCS:GunSteVei24}.}
    \label{tab:obfuscation-summary}
\end{table}

\section{Obfuscating Classic McEliece with TODO} \label{sec:obfuscating-classic-mceliece}

\paragraph{Original Generation and Distribution}
Classic McEliece \cite{NISTPQC-R4:ClassicMcEliece22} is a code-based KEM in round 4 of the NIST PQC standardization. We only consider the so-called "non-f" versions of the specified parameter sets which are interoperable with the f versions, and have simpler but slower key generation. TODO: OK? Keita also makes that assumption at some point.

In Classic McEliece, public keys are the rightmost column of a generator matrix in systematic form. Ciphertexts are the data bits corresponding to codewords with Hamming weight exactly $t$.

The generation of public keys occurs in a multistep process starting from a 256-bit seed which is expanded using SHAKE256 into a secondary seed and raw values. If the raw values do not form a suitable systematic code, another attempt is started from the secondary seed, and so on.

For encapsulation, rejection-sampling is used to generate a uniformly random vector of Hamming weight $t$ which is then mapped from the space of $n$-bit vectors to $mt=n-k$ bits.

We define the same assumptions about computational hardness as in \cite[Definition~K.1]{EC:Xagawa22}. \cref{fig:classic-mceliece-assumptions} shows corresponding game-based definitions.
\begin{itemize}
    \item \textbf{PR-Key assumption:} It is computationally hard to distinguish real public keys from the rightmost columns of the systematic forms of uniformly random generator matrices (conditional on the systematic form existing).
    \item \textbf{modified Decisional Syndrome Decoding assumption:} It is computationally hard to distinguish real ciphertexts from uniformly random $(n-k)$-bit vectors.
\end{itemize}

For both games $\textsf{goal} \in \{\prkey, \mdsd\}$, we define the advantage of an adversary $\adv$ as follows and further say that Classic McEliece is $(t, \epsilon)\textsf{-goal}$-secure if for any adversary $\adv$ with running time at most $t$, we have that:
\[ \advantage{\textsf{goal}}{}[(\adv)] := 2 \cdot \Pr[\textsf{goal}(\adv) = 1] - 1 \leq \epsilon \]

\begin{figure}
    \begin{pchstack}[boxed, center, space=0.5em]

\begin{pcvstack}[space=0.05em]
\procedure[linenumbering]{$\textbf{GAME } \textsf{PR-Key}(\adv)$}{
    b \gets \bin \\
    (T_0,\sk) \gets \kgen() \\
    T_1 \gets \textsf{RandGen}() \\
    b' \gets \adv(T_b) \\
    \pcreturn \llbracket b' = b \rrbracket
}

\procedure[lnstart=5,linenumbering]{$\textbf{GAME } \textsf{mDSD}(\adv)$}{
    b \gets \bin \\
    \hat T \gets \textsf{RandGen}() \\
    e \gets \textsf{FixedWeight}() \\
    u_0 \gets \left[ I_{mt} \mid \hat T \right] \cdot e \\
    u_1 \gets U(\mathbb F_2^{mt}) \\
    b' \gets \adv(\hat T, u_b) \\
    \pcreturn \llbracket b' = b \rrbracket
}
\end{pcvstack}

\begin{pcvstack}[space=0.05em]
\procedure[lnstart=12,linenumbering]{$\textsf{RandGen}()$}{
    \hat T \gets \bot \\
    \pcwhile \hat T = \bot \pcdo \\
    \t  \hat H \gets U(\mathbb F_2^{mt \times n}) \\
    \t  \text{reduce } \hat H \text{ to systematic form } \left[ I_{mt} \mid \hat T \right] \\
    \label{ln:randgen-reject}
    \t  \text{if this fails } \hat T \gets \bot \\ 
    \pcendwhile \\
    \pcreturn \hat T
}

\procedure[lnstart=19,linenumbering]{$\textsf{RandGen}'()$}{
    \hat T \gets U(\mathbb F_2^{mt \times k}) \\ 
    \pcreturn \hat T
}
\end{pcvstack}

\end{pchstack}

    \caption{Indistinguishability games for Classic McEliece with $\kgen$ and \textsf{FixedWeight} as defined in \cite{NISTPQC-R4:ClassicMcEliece22}. These follow the definitions from \cite[Definition~K.1]{EC:Xagawa22}.}
    \label{fig:classic-mceliece-assumptions}
\end{figure}

\paragraph{Constructed Encoding}

Public keys ... ? TODO

Ciphertexts are already hard to distinguish from uniform random bit vectors following the $\mdsd$ assumption. We therefore only need to take care to pad the bit vectors up to the byte boundary with randomly sampled bits that can be discarded during decoding.

\cref{fig:classic-mceliece-encoding} shows the encoding functions for TODO TBD in pseudocode.

\begin{figure}
    \input{algorithms/classic-mceliece-encoding}
    \caption{Encoding functions for Classic McEliece.}
    \label{fig:classic-mceliece-encoding}
\end{figure}

\paragraph{Uniformity and Success Rate}

We analyze the public key and ciphertext uniformity of our encoding and reduce to the two assumptions made above. We analyze the success rate of our encoding.

\begin{lemma}[Public key uniformity of TODO TBD] \label{lem:classic-mceliece-pk-unif}
    TODO bounds on uniformity
\end{lemma}

\begin{lemma}[Ciphertext uniformity of TODO TBD] \label{lem:classic-mceliece-ctxt-unif}
    TODO bounds on uniformity
\end{lemma}

\begin{lemma}[First-keygen success probability of TODO TBD] \label{lem:classic-mceliece-first-keygen-success}
    TODO bounds on success
\end{lemma}

\begin{lemma}[[First-encap success probability of TODO TBD] \label{lem:classic-mceliece-first-encap-success}
    TODO bounds on success
\end{lemma}

Together with the Classic McEliece KEM, our encoding functions can be used to define a keygen/encapsulate-then-encode obfuscated KEM as in \cref{def:keygen-then-encode} and \ref{def:keygen-encap-then-encode}.

\begin{theorem}
    Let TODO TBD be a keygen/encapsulate-then-encode obfuscated KEM based on the Classic McEliece KEM as per \cref{def:keygen-then-encode} and \ref{def:keygen-encap-then-encode}. For any adversary $\adv$ against the $\indcca$ security of TODO TBD, there exists an algorithm $\bdv$ such that
    \[ \advantage{\indcca}{TODO TBD}[(\bdv)] \leq 123 \]

    Further,  For any adversary $\adv$ against the $\sprcca$ security of TODO TBD, there exists an algorithm $\bdv$ such that
    \[ \advantage{\sprcca}{TODO TBD}[(\bdv)] \leq 123 \]
\end{theorem}
\begin{proof}
    IND-CCA security follows from \cite[Theorem~2.12]{CCS:GunSteVei24}, together with \cref{lem:classic-mceliece-first-keygen-success} and \ref{lem:classic-mceliece-first-encap-success}.
    
    SPR-CCA security follows from \cite[Theorem~2.13]{CCS:GunSteVei24}, together with \cref{lem:classic-mceliece-first-keygen-success}, \ref{lem:classic-mceliece-first-encap-success} and \ref{lem:classic-mceliece-ctxt-unif}.
\end{proof}

\section{Obfuscating ABC with XYZ} \label{sec:tbd}

