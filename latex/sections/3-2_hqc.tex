\section{Obfuscating HQC} \label{sec:obfuscating-hqc}

Throughout this \cref{sec:obfuscating-hqc}, we use the following variables as defined in as defined in \cite{NISTPQC-R4:HQC22}: the integer variables $n, n_1, n_2$, denoting the length of employed codes, as well as the integer variables $w$, and $w_e = w_r$, as the hamming weights of randomly sampled vectors. Further, we use $k = 512$ as the bitlength of the shared secret in all HQC parameter sets as in \cite[Figure~3, Table~7]{NISTPQC-R4:HQC22}.
As in \cref{sec:obfuscating-classic-mceliece}, we work with polynomials and vectors over $\mathbb F_2$ when examining the KEM and use bit strings in the definition and analysis of OKEMs. As the degrees of the polynomials and the dimensions of the vectors are fixed for a given parameter set, we associate them with their canonical encoding as bit strings of appropriate length.

Let $\mathcal R \subset \mathbb{F}_2[x]$ be the ring of polynomials over $\mathbb F_2$ with degree $<n$, specifically $\mathcal R := \mathbb{F}_2[x]/(x^n-1)$. Let $\mathcal R_w \subset \mathcal R$ denote the subset of polynomials with hamming weight exactly $w$. The notion $\mathbf u(1)$ is used to denote the evaluation of a polynomial $\mathbf u(x) \in \mathbb{F}_2[x]$ at the point $x=1$, note that $\mathbf u(x) \in \mathbb{F}_2$ for any fixed value of $x \in \mathbb{F}_2$. To refer to the coefficients of $\mathbf u$, we write $\mathbf u_i \in \mathbb{F}_2$, such that $\mathbf u(x) = \sum_{i=0}^{n-1} \mathbf u_i \cdot x^i$.

\paragraph{Original generation and distribution of $\mathsf{HQC}$}

HQC is a code-based KEM selected for standardization as part of the NIST Post-Quantum Cryptography Standardization \cite{nist-standardization,nist-ir-8545}. HQC consists of three parameter sets, namely HQC-128, HQC-192, and HQC-256.

In HQC, public keys are pairs of polynomials $(\mathbf h, \mathbf s) \in \mathcal R$ such that $\mathbf s = \mathbf x + \mathbf h \mathbf y$ where $\mathbf x,\mathbf y,\mathbf h$ have degree $<n$ and $x,y$ have hamming weight exactly $w$.
The HQC ciphertexts are triples $\mathbf u,\mathbf v,\mathit{salt}$ where $\mathbf u$ is a polynomial of degree $<n$, $\mathbf v$ is a bit string of length $n_1n_2$, and $\mathit{salt}$ is a uniformly random 128-bit string used to derive randomness.

The processes for key generation and encapsulation are reproduced from the Round 4 specification \cite{NISTPQC-R4:HQC22} in \cref{fig:hqc-spec}. That is, public keys are generated by directly sampling from appropriate distributions, and encapsulation involves fixing the encryption randomness using $\mathit{salt}$, before encrypting a uniformly random message and deriving the shared secret from the message and the ciphertext using a random oracle.

\begin{figure}
    \begin{pchstack}[boxed, center, space=0.5em]

\procedure[linenumbering]{$\kgen()$ \titlecomment{\text{see \cite[Figure~3]{NISTPQC-R4:HQC22}}}}{
    \mathbf h \gets \mathcal R \\
    \sigma \getsr \mathbb F_2^k \\
    \mathbf x, \mathbf y \getsr \mathcal R_w^2 \\
    \mathbf s \gets \mathbf x + \mathbf h \cdot \mathbf y \\
    \pcreturn \pk=(\mathbf h, \mathbf s), \sk=(\mathbf x, \mathbf y, \sigma)
}

\procedure[lnstart=5,linenumbering]{$\encaps(\pk=(\mathbf h, \mathbf s))$ \titlecomment{\text{see \cite[Figure~3]{NISTPQC-R4:HQC22}}}}{
    \mathbf m \getsr \mathbb F_2^k \\
    \mathit{salt} \getsr \mathbb F_2^k \\
    \theta \gets \mathcal G(\mathbf m \mathbin\Vert \mathsf{firstBytes}(\pk,32) \mathbin\Vert \mathit{salt}) \\
    \mathbf e, \mathbf r_1, \mathbf r_2 \gets \text{generated from randomness } \theta \pcskipln \\
    \t  \text{such that } w(\mathbf e) = w_e \pcskipln \\
    \t  \text{and } w(\mathbf r_1) = w(\mathbf r_2) = w_r \\
    \mathbf u \gets \mathbf r_1 + \mathbf h \cdot \mathbf r_2 \\
    \mathbf v \gets \mathsf{truncate}(\mathbf m \mathbf G + \mathbf s \cdot \mathbf r_2 + \mathbf e, \ell) \\
    \mathbf c \gets (\mathbf u, \mathbf v) \\
    K \gets \mathcal{K}(\mathbf m, \mathbf c) \\
    \pcreturn ((\mathbf c, \mathit{salt}), K)
}

\end{pchstack}

    \caption[
        A selection of algorithms from the HQC Round 4 specification.
    ]{
        A relevant selection of algorithms from the HQC Round 4 specification \cite{NISTPQC-R4:HQC22}. $\ell$ is defined to be $=n-n_1n_2$. $\mathcal G, \mathcal K$ refer to hash functions which we model as random oracles. $w(h)$ denotes the hamming weight of a vector or polynomial $h$. The generator matrix $\mathbf G$ is part of the parameter sets and statically known.
        For the sake of brevity, decapsulation and the functions $\mathsf{truncate}()$ (removing a specified number of bits) and $\mathsf{firstBytes}()$ (selecting a number of bytes) are omitted here.}
    \label{fig:hqc-spec}
\end{figure}

It has been shown in \cite[Lemma~P.2, Theorem~P.1]{EC:Xagawa22} that HQC-192's $\sprcca$ security (with respect to a simulator that samples $\mathbf u,\mathbf v$ uniformly at random) reduces to the following two assumptions:
\begin{itemize}
    \item the 2-Decisional Quasi-Cyclic Syndrome Decoding ($\twodqcsd$) assumption \cite[Definition~2.1.15]{NISTPQC-R4:HQC22}
    \item the 3-Decisional Quasi-Cyclic Syndrome Decoding ($\threedqcsd$) assumption \cite[Definition~2.1.17]{NISTPQC-R4:HQC22}
\end{itemize}
When referring to these assumptions in the following, we will implicitly use the parameters: $b_1 = 1$ as the public key parity, as well as $b_2 = (1 + b_1) \cdot w \mod 2, b_3 = (1+b_1) \cdot w_r$ and $\ell = n - n_1n_2$.

A notable detail is that while \cite{EC:Xagawa22} does not mention $\mathit{salt}$, the proofs still apply for the first two ciphertext components $\mathbf u, \mathbf v$. The $\mathit{salt}$ is by definition a uniformly random value.

\paragraph{Constructed encoding $E_\mathsf{HQC}$}

The two latter components of HQC ciphertexts $\mathbf v, \mathit{salt}$ are indistinguishable from random bit strings due to \cite{EC:Xagawa22} and by definition, respectively. We therefore need only be concerned with obfuscating $\mathbf u$.

As shown in \cite{EC:Xagawa22}, the parity of $\mathbf u$ depends on the public key parity as $\mathbf u(1)=(1 + \mathbf h(1))\cdot w_r \mod 2$. For HQC-192, $w_r = 0 \mod 2$ and ciphertexts therefore always have parity $\mathbf u(1)=0$. For HQC-128 and HQC-256, $w_r = 1 \mod 2$ and the parity of ciphertexts depends on the parity of the public key component as $\mathbf u(1)=1 + \mathbf h(1)$.

Recall that our goal is to encode HQC ciphertexts as bit strings indistinguishable from uniformly random bit strings. Thus, when encoding a ciphertext for HQC-192, we remove the last bit of $\mathbf u$, i.e. $\mathbf u_{n-1}$, canonically encode the rest of $\mathbf u$ to a bit string, and pad with random bits up to the byte boundary. The other parts of the ciphertext align with the byte boundary due to their size. The single dropped bit of $\mathbf u$ can be reconstructed uniquely during decoding such that the parity of $\mathbf u(1)=0$ is achieved. This is because $\mathbf u(1)=0 \iff \bigoplus_{i=0}^{n-1} \mathbf u_i = 0 \iff \mathbf u_{n-1} = \bigoplus_{i=0}^{n-2} \mathbf u_i$.

For the two remaining parameter sets, HQC-128 and HQC-256, the only obstacle in the $\sprcca$ proof of \cite{EC:Xagawa22} is that the $\mathbf u$ component leaks the parity of the public key component $\mathbf h$ and a good simulator would have to know the parity of the public key component, which was not needed for HQC-192.
We can circumvent this issue by rejecting half of all public keys (specifically those with parity $\mathbf h(1)=0$) during key generation for HQC-128 and HQC-256.
With this change, the polynomial $\mathbf u$ in the ciphertexts now has parity $\mathbf u(1)=0$ for all three parameter sets and is indistinguishable from uniformly randomly sampled 0-parity polynomials. It can then be encoded just as for HQC-192 ciphertexts (that is, removing the last bit and padding to the byte boundary).

The padding with random bits requires (after dropping one bit) the addition of 6, 4, and 6 random bits for the parameter sets HQC-128, HQC-192, and HQC-256, respectively. We omit this detail here because the $\feo$ construction works on bit strings of arbitrary size, but it remains relevant for implementation.

\Cref{fig:hqc-encoding} shows pseudocode for the encoding mentioned above, which we will call $E_\mathsf{HQC}$. An OKEM based on HQC can be constructed as $\feo[\mathsf{HQC}, E_\mathsf{HQC}]$ as in \cref{def:encaps-then-encode}.

\begin{figure}
    \begin{pcvstack}[boxed, center, space=0.05em]

\pchspace

\centerline{\textbf{Public Key Filtering:}}

\begin{pchstack}[space=0.5em]
\procedure[linenumbering]{$E_\mathsf{HQC}.\textsf{Filter}(\pk=(\mathbf h,\mathbf s))$}{
    \pcreturn \pctrue \pccomment{HQC-192} \\
    p \gets \mathbf h(1) \pccomment{$p \in \mathbb F_2$} \\
    \pcreturn \llbracket p = 1 \rrbracket \pccomment{HQC-128, HQC-256}
}
\end{pchstack}

\centerline{\textbf{Ciphertext Encoding/Decoding:}}

\begin{pchstack}[space=0.5em]
\procedure[lnstart=3,linenumbering]{$E_\mathsf{HQC}.\encodectxt(c)$}{
    (\mathbf u, \mathbf v, \mathit{salt}) \gets c \\
    \pcreturn \textsf{truncate}(\mathbf u, 1) \mathbin\Vert \mathbf v \mathbin\Vert \mathit{salt}
}

\procedure[lnstart=5,linenumbering]{$E_\mathsf{HQC}.\decodectxt(\hat c)$}{
    \hat{\mathbf u} \mathbin\Vert \mathbf v \mathbin\Vert \mathit{salt} \gets \hat{c} \\
    b \gets \bigoplus_{i=0}^{n-2} \hat{\mathbf u}_i \pccomment{$b \in \mathbb F_2$} \\
    \mathbf u(x) \gets \hat{\mathbf u}(x) + b \cdot x^{n-1} \\
    \pcreturn (\mathbf u, \mathbf v, \mathit{salt})
}
\end{pchstack}

\end{pcvstack}

    \caption[
        Our filter-encode obfuscator $E_\mathsf{HQC}$ for obfuscating all HQC parameter sets.
    ]{
        Our filter-encode obfuscator $E_\mathsf{HQC}$ for obfuscating all HQC parameter sets. It rejects all public keys with parity $\mathbf h(1)=0$ for the parameter sets HQC-128 and HQC-256. For HQC-192, no public keys are rejected. This ensures that $\mathbf u(1)=0$ allowing us to drop one redundant bit during encoding and reconstruct it when decoding. As in \cref{fig:hqc-spec}, the function $\mathsf{truncate}()$ denotes the removal of a specified number of bits from the end of a bit string.
    }
    \label{fig:hqc-encoding}
\end{figure}

\paragraph{Uniformity and success rate of $\feo[\mathsf{HQC}, E_\mathsf{HQC}]$}

We analyze the ciphertext uniformity of our encoding and reduce it to the two assumptions made above. We analyze the success rate of our encoding.

In the following, let $\mathsf{HQC}$ denote the HQC KEM as defined in \cite{NISTPQC-R4:HQC22} and \cref{fig:hqc-spec} using any of the three parameter sets HQC-128, HQC-192, or HQC-256.

\begin{lemma}[First-keygen Success Probability]
\label{lem:hqc-first-keygen-success}
    The first-keygen success probability of $\feo[\mathsf{HQC}, E_\mathsf{HQC}]$ is:
    \begin{itemize}
        \item $\epsilon^\firstkeygensuccess_{\mathsf{HQC}, E_\mathsf{HQC}} = 1$ for the parameter set HQC-192
        \item $\epsilon^\firstkeygensuccess_{\mathsf{HQC}, E_\mathsf{HQC}} = \frac12$ for the parameter sets HQC-128, and HQC-256.
    \end{itemize}
\end{lemma}
\begin{proof}
    The first statement follows trivially from the definition of $E_\mathsf{HQC}$ in \cref{fig:hqc-encoding}.
    For the latter point, we also use that the definition of HQC in \cref{fig:hqc-spec} samples $\mathbf h$ uniformly at random, and therefore the parity is uniformly distributed over $\bin$.
\end{proof}

\begin{lemma}[First-encaps Success Probability]
\label{lem:hqc-first-encaps-success}
    The first-encaps success probability of $\feo[\mathsf{HQC}, E_\mathsf{HQC}]$ is
    \[ \epsilon^\firstencapssuccess_{\mathsf{HQC}, E_\mathsf{HQC}} = 1. \]
\end{lemma}

\begin{lemma}[Ciphertext Uniformity]
\label{lem:hqc-ctxt-unif}
    Let $\mathbf F_0 := \{u \in \mathcal R \mid u(1) = 0 \}$ be the set of 0-parity polynomials with degree $<n$.
    Let $\mathcal S$ be a KEM encapsulation simulator that outputs $\mathbf u \getsr \mathcal F_0, \mathbf v \getsr \bin^{n_1n_2}, \mathit{salt} \getsr \bin^{128}$.
    For brevity, let further $\OKEM := \feo[\mathsf{HQC}, E_\mathsf{HQC}]$ and $\sigma := \epsilon^\firstkeygensuccess_{\mathsf{HQC}, E_\mathsf{HQC}}$ as per \cref{lem:hqc-first-keygen-success}. 

    For any adversary $\adv$ against the ciphertext uniformity of $\OKEM$, there exists an adversary $\bdv$ against the $\sprcca$ security of $\mathsf{HQC}$ with respect to the simulator $\mathcal S$, and adversaries $\cdv_1, \cdv_2$ against the $\twodqcsd$ and $\threedqcsd$ assumptions respectively, such that
    \[
        \advantage{\ctxtunif}{\OKEM}[(\adv)]
        \leq 1/\sigma \advantage{\sprcca}{\mathsf{HQC},\mathcal S}[(\bdv)]
        \leq 1/\sigma \left(
            \advantage{\twodqcsd}{}[(\cdv_1)] + \advantage{\threedqcsd}{}[(\cdv_2)]
        \right).
    \]
\end{lemma}
\begin{proof}
    The encoding that we defined outputs bit strings of size $\obfctxtlen := n-1+n_1n_2+128$.

    The reduction from ciphertext uniformity to $\sprcca$ follows directly from \cref{lem:ctxt-unif-for-bijections} using the function $E_\mathsf{HQC}.\encodectxt: \mathcal F_0 \to \bin^\obfctxtlen$, mapping vectors and polynomials to their canonical representation as bit strings.
    This encoding function is deterministic, bijective, and never fails, that is, $\epsilon^\firstencapssuccess_{\mathsf{HQC},\encodectxt} = 1$ according to \cref{lem:hqc-first-encaps-success}.

    Recall also that the first-keygen success probability depends on the parameter set, this is reflected in $\sigma$.

    In \cite[Lemma~P.2]{EC:Xagawa22}, a reduction is given from $\sprcca$ w.r.t. $\mathcal S$ to the $\twodqcsd$ and $\threedqcsd$ assumptions. The proof can easily be adapted to include the $\mathit{salt}$ which would increase the number of arguments for the random oracle $\mathsf G$, but no additional assumption is needed.
\end{proof}

The OKEM construction also satisfies the KEM security notions.

\begin{theorem}[KEM Security]
    Let $\OKEM := \feo[\mathsf{HQC}, E_\mathsf{HQC}]$ be a filter-encode obfuscated KEM as per \cref{def:encaps-then-encode}, and let $\sigma := \epsilon^\firstkeygensuccess_{\mathsf{HQC}, E_\mathsf{HQC}}$ as per \cref{lem:hqc-first-keygen-success}.
    
    For any adversary $\adv$ against the $\indcca$ security of $\OKEM$, there exists an algorithm $\bdv$ such that
    \[ \advantage{\indcca}{\OKEM}[(\adv)] \leq 1/\sigma \cdot \advantage{\indcca}{\mathsf{HQC}}[(\bdv)]. \]

    Further, for any adversary $\adv$ against the $\sprcca$ security of $\OKEM$, there exist algorithms $\bdv_1, \bdv_2$ and $\cdv_1, \cdv_2$ such that
    \begin{align*}
        \advantage{\sprcca}{\OKEM}[(\adv)] &\leq
        1/\sigma \cdot \advantage{\sprcca}{\mathsf{HQC},\simulunif}[(\bdv_1)]
        + \advantage{\ctxtunif}{\OKEM}[(\bdv_2)] \\
        &\leq
        1/\sigma \left(
            \advantage{\sprcca}{\mathsf{HQC},\simulunif}[(\bdv_1)]
            + \advantage{\twodqcsd}{}[(\cdv_1)] + \advantage{\threedqcsd}{}[(\cdv_2)]
        \right).
    \end{align*}
\end{theorem}
\begin{proof}
    IND-CCA security follows from \cite[Theorem~2.12]{CCS:GunSteVei24}, together with \cref{lem:hqc-first-keygen-success,lem:hqc-first-encaps-success}.
    
    SPR-CCA security follows from \cite[Theorem~2.13]{CCS:GunSteVei24}, together with \cref{lem:hqc-first-keygen-success,lem:hqc-first-encaps-success,lem:hqc-ctxt-unif}.
\end{proof}