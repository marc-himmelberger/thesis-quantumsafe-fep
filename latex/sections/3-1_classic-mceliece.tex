\section{Classic McEliece as an obfuscated KEM} \label{sec:obfuscating-classic-mceliece}

Throughout this \cref{sec:obfuscating-classic-mceliece}, we use the positive integer variables $m,n,t,k$ where $n < 2^m$ and $k=mt-n$ as defined in \cite{NISTPQC-R4:ClassicMcEliece22}. We work with matrices and vectors over $\mathbb F_2$ when examining the KEM and use bitstrings in the defintion and analysis of OKEMs. As the dimensions of the matrices and vectors are fixed for a given parameter set, we associate the matrices and vectors with their canonical encoding as bitstrings of appropriate length.

\paragraph{Original Generation and Distribution}
Classic McEliece \cite{NISTPQC-R4:ClassicMcEliece22} is a perfectly correct, code-based KEM in round 4 of the NIST PQC standardization. We only consider the so-called "non-f" versions of the specified parameter sets which are interoperable with the f versions, and have simpler but slower key generation.

In Classic McEliece, public keys are the $k$ rightmost columns of a $mt \times n$ generator matrix in systematic form ($k = n - mt$). Ciphertexts are the $mt$ data bits corresponding to $n$-bit codewords with Hamming weight exactly $t$.

The generation of public keys occurs in a multistep process starting from a 256-bit seed which is expanded using SHAKE256 into a secondary seed and raw values. If the raw values do not form a suitable systematic code, another attempt is started from the secondary seed, and so on. This key generation process, among other algorithms from the Round 4 specification of Classic McEliece are reproduced in \cref{fig:classic-mceliece-spec}.

For encapsulation, rejection-sampling is used to generate a uniformly random vector of Hamming weight exactly $t$ which is then mapped from the space of $n$-bit vectors to $mt=n-k$ bits using the generator matrix defined by the public key.

\begin{figure}
    \begin{pchstack}[boxed, center, space=0.5em]

\begin{pcvstack}[space=0.05em]
\procedure[linenumbering]{$\kgen()$ \titlecomment{\text{see \cite[Sec. 5.3]{NISTPQC-R4:ClassicMcEliece22}}}}{
    \delta \gets \bin^\ell \\
    X, \delta' \gets G(\delta) \pccomment{$\delta' \in \bin^\ell$} \\
    \text{if anything fails, restart with } \delta=\delta' \\
    s, \alpha_0, \dots, \alpha_{q-1}, g \gets X \pcskipln \\
    \t  \text{where $s$ is an $n$-bit string,}  \pcskipln \\
    \t  \text{$\alpha_0, \dots, \alpha_{q-1}$ are distrinct elements of $\mathbb F_q$,}  \pcskipln \\
    \t  \text{$g \in \mathbb F_q[x]$ is monic, irreducible,} \pcskipln \\
    \t  \text{and $\deg(g)=t$} \\
    \Gamma \gets (g, \alpha_0, \dots, \alpha_{n-1}) \\
    T \gets \textsf{MatGen}(\Gamma) \\
    \pcreturn \pk=T, \sk=(\delta, g, \alpha_0, \dots, \alpha_{q-1}, s)
}
\procedure[lnstart=7,linenumbering]{$\textsf{MatGen}(\Gamma=(g, \alpha_0, \dots, \alpha_{n-1}))$ \kern-1em \titlecomment{\text{see \cite[Sec. 4.2]{NISTPQC-R4:ClassicMcEliece22}}}}{
    \tilde H \gets \left\{\alpha_j^i/g(\alpha_j)\right\}_{i,j} \pccomment{$\tilde H \in \mathbb F_q^{t \times n}$}\\
    \hat H \gets \text{expand each polynomial entry of } \tilde H \pcskipln \\
    \text{ into $m$ rows of bits denoting its coefficients} \\
    (I_{mt} \mid T) \gets \text{reduce } \hat H \text{ into systematic form} \\
    \text{if this fails } T \gets \bot \\ 
    \pcreturn T
}
\end{pcvstack}

\begin{pcvstack}[space=0.05em]
\procedure[lnstart=12,linenumbering]{$\encaps()$ \titlecomment{\text{see \cite[Sec. 4.3]{NISTPQC-R4:ClassicMcEliece22}}}}{
    e \gets \textsf{FixedWeight}() \\
    H \gets (I_{mt} \mid T) \\
    C \gets He \in \mathbb F_2^{mt} \\
    K \gets H(1,e,C) \\
    \pcreturn (C, K)
}

\procedure[lnstart=17,linenumbering]{$\textsf{FixedWeight}()$ \titlecomment{\text{see \cite[Sec. 5.4]{NISTPQC-R4:ClassicMcEliece22}}}}{
    a_0, \dots, a_{t-1} \gets [0, n-1] \\
    \text{if not all $a_i$ are distrinct, restart} \\
    e \gets (e_0, \dots, e_{n-1}) \pcskipln \\
    \t  \text{ where, for all } i: e_{a_i} = 1 \\
    \pcreturn e
}
\end{pcvstack}

\end{pchstack}

    \caption{A relevant selection of algorithms from the Classic McEliece Round 4 specification \cite{NISTPQC-R4:ClassicMcEliece22}. $q=2^m$ refers to the size of a polynomial field of characteristic $2$. $G$ refers to a pseudorandom bit generator mapping a string of $\ell$ bits to a string of $\geq n + mq + mt + \ell$ bits. For the sake of brevity, the many more technical details, as well as the semi-systematic form, are omitted.}
    \label{fig:classic-mceliece-spec}
\end{figure}

We define the same assumptions about computational hardness as in \cite[Definition~K.1]{EC:Xagawa22}. These have appeared in more general forms in previous literature studying the Niederreiter and McEliece cryptosystems, e.g. \cite{AC:CouFinSen01,EC:SaiXagYam18}. \cref{fig:classic-mceliece-assumptions} shows corresponding game-based definitions.
\begin{itemize}
    \item \textbf{modified PR-Key assumption:} It is computationally hard to distinguish real public keys from the rightmost columns of the systematic forms of uniformly random generator matrices (conditioned on the systematic form existing).
    \item \textbf{modified Decisional Syndrome Decoding assumption:} It is computationally hard to distinguish ciphertexts generated using uniformly random generator matrices from uniformly random $(n-k)$-bit vectors.
\end{itemize}

For both games $\textsf{goal} \in \{\prkey, \mdsd\}$, we define the advantage of an adversary $\adv$ as follows and further say that Classic McEliece is $(t, \epsilon)\textsf{-goal}$-secure if for any adversary $\adv$ with running time at most $t$, we have that:
\[ \advantage{\textsf{goal}}{}[(\adv)] := 2 \left| \Pr[\textsf{goal}(\adv) = 1] - \frac{1}{2} \right| \leq \epsilon \]

\begin{figure}
    \begin{pchstack}[boxed, center, space=0.5em]

\begin{pcvstack}[space=0.05em]
\procedure[linenumbering]{$\textbf{GAME } \textsf{PR-Key}(\adv)$}{
    b \gets \bin \\
    (T_0,\sk) \gets \kgen() \\
    T_1 \gets \textsf{RandGen}() \\
    b' \gets \adv(T_b) \\
    \pcreturn \llbracket b' = b \rrbracket
}

\procedure[lnstart=5,linenumbering]{$\textbf{GAME } \textsf{mDSD}(\adv)$}{
    b \gets \bin \\
    \hat T \gets \textsf{RandGen}() \\
    e \gets \textsf{FixedWeight}() \\
    u_0 \gets \left[ I_{mt} \mid \hat T \right] \cdot e \\
    u_1 \gets U(\mathbb F_2^{mt}) \\
    b' \gets \adv(\hat T, u_b) \\
    \pcreturn \llbracket b' = b \rrbracket
}
\end{pcvstack}

\begin{pcvstack}[space=0.05em]
\procedure[lnstart=12,linenumbering]{$\textsf{RandGen}()$}{
    \hat T \gets \bot \\
    \pcwhile \hat T = \bot \pcdo \\
    \t  \hat H \gets U(\mathbb F_2^{mt \times n}) \\
    \t  \text{reduce } \hat H \text{ to systematic form } \left[ I_{mt} \mid \hat T \right] \\
    \label{ln:randgen-reject}
    \t  \text{if this fails } \hat T \gets \bot \\ 
    \pcendwhile \\
    \pcreturn \hat T
}

\procedure[lnstart=19,linenumbering]{$\textsf{RandGen}'()$}{
    \hat T \gets U(\mathbb F_2^{mt \times k}) \\ 
    \pcreturn \hat T
}
\end{pcvstack}

\end{pchstack}

    \caption{Indistinguishability games for Classic McEliece with $\kgen$ and \textsf{FixedWeight} as defined in \cref{fig:classic-mceliece-spec}. These games follow the definitions from \cite[Definition~K.1]{EC:Xagawa22}. Note that $n-k = mt$.}
    \label{fig:classic-mceliece-assumptions}
\end{figure}

\paragraph{Classic McEliece is already obfuscated}

Given the $\prkey$ assumption, public keys are hard to distinguish from outputs of $\textsf{RandGen}()$. We claim that $\textsf{RandGen}()$ can be simplified to a new algorithm $\textsf{RandGen}'()$, also shown in \cref{fig:classic-mceliece-assumptions}, with statistical indistinguishability.

\begin{lemma}[Simplifying \textsf{RandGen}] \label{lem:classic-mceliece-randgen-prime}
    Let $\textsf{RandGen}(), \textsf{RandGen}'()$ be defined as in \cref{fig:classic-mceliece-assumptions}.
    The distributions $D_0 = \{ \hat T \gets \textsf{RandGen}() \}$ and $D_1 = \{ \hat T \gets \textsf{RandGen}'() \}$ are statistically indistinguishable, i.e. have statistical distance $\Delta(D_0, D_1) = 0$.
\end{lemma}

\begin{proof}
    We first examine the rejection sampling behavior around \cref{ln:randgen-reject}: A resampling occurs if and only if a reduction of $\hat H$ to systematic form (also known as \emph{reduced row echelon form} or \emph{row canonical form}) fails. This reduction, in turn, fails if and only if the $mt$ leftmost columns of $\hat H$ do not form a full rank square matrix.
    The probability for a $mt \times mt$ matrix with i.i.d. uniformly random entries over $\mathbb F_2$ to be of full rank is exactly \cite{DBLP:journals/corr/SalmondGGC14}
    \[ p_\textsf{FR} := \prod_{i=1}^{mt} \left( 1-2^{-i} \right) \]
    
    For the parameter sets used in Classic McEliece ($768 \leq mt \leq 1664$), this calculates to \[ p_\textsf{FR} \approx 0.288788 \]
    Differences between the parameter sets are negligible and a strict upper bound of $p_\textsf{FR} \leq \prod_{i=1}^{768} \left( 1-2^{-i} \right) \leq \prod_{i=1}^{8} \left( 1-2^{-i} \right) \leq 0.29$ applies. A proof for a tight lower bound eludes us although the numerical convergence is rapid.
    
    This calculation validates our analysis, as the resulting probability is the same as was experimentally reported in \cite[security.pdf: Section 4.2]{NISTPQC-R4:ClassicMcEliece22}.
    
    We now consider the possible influence that the reduction to systematic form might have on the distribution of bits in the simulated public key $\hat T$.

    Consider an intermediate value of $\hat H$ which can be reduced to systematic form (i.e. has full rank and will pass the check on \cref{ln:randgen-reject}). This value exists precisely once for all terminating executions of $\textsf{RandGen}()$ as the last $\hat H$ sampled before returning a result.

    In general, the reduced row echelon form can be obtained using Gauss–Jordan elimination. This algorithm carries out a sequence of operations on the matrix consisting of swapping rows, adding rows onto others and scaling rows.
    Over $\mathbb F_2$, the scaling of rows is not needed and swapping of rows can be accomplished via a sequence of 3 row additions. Any Gauss-Jordan reduction to systematic form can thus be expressed as merely a sequence of row additions.
    The systematic form of $\hat H$ over $\mathbb F_2$, just like reduced row echelon forms over any field, is also unique. Thus it suffices to consider an arbitrary sequence of row additions transforming $\hat H$ to systematic form.
    
    Due to the number of rows, a reduction is possible in $(mt)^2$ row additions: The diagonal can be filled with ones using at most $mt$ additions. After another $(mt)^2-mt$ additions at the latest, all other entries in the leftmost $mt$ columns are zeroes. To determine one such order of row additions which bring $\hat H$ into systematic form, it suffices to consider only the leftmost $mt$ columns.
    
    Let $M_0$ denote the rightmost columns of $\hat H$, which form (by definition) a $mt \times k$ matrix of uniformly random i.i.d. entries from $\mathbb F_2$, in particular, independent from the leftmost columns of $\hat H$.
    
    As the addition of any one row of uniformly random bits onto any other row of independent uniformly random bits produces a row of uniformly random i.i.d. bits, the resulting matrix after one row addition, say $M_1$, is statistically indistinguishable from $M_0$.
    
    Repeating this argument for a maximum of $(mt)^2$ times, we can conclude that $\hat T$ is statistically indistinguishable from a uniformly random $mt \times k$ matrix over $\mathbb F_2$ as in $\textsf{RandGen}'()$.
\end{proof}

Ciphertexts are also statistically indistinguishable from uniform random bit vectors following the $\prkey$ and $\mdsd$ assumptions. Implementers should note that padding public keys and ciphertexts to the byte boundary may require \emph{randomized} padding. This is only needed if the number of bits is not evenly divisible by 8, and occurs only for the parameter set 'mceliece6960119'.  All other parameter sets have $mt = 0 \mod 8$. Padding of these values with zero bits (as suggested in the specification) would break the public key and ciphertext uniformity.

\begin{theorem}
    Classic McEliece, as specified in \cite{NISTPQC-R4:ClassicMcEliece22}, is an obfuscated KEM without requiring any special encoding of public keys or ciphertexts.

    Classic McEliece, as an OKEM, achieves $\indcca$ and $\sprcca$ security according to \cref{def:ind-spr-cca}.
\end{theorem}
\begin{proof}
    TODO: discuss syntax of OKEM vs KEM

    The $\indcca$ and $\sprcca$ security notions for OKEMs defined in \cref{def:ind-spr-cca} match the established notions of the same names. These have been shown for Classic McEliece in previous work \cite{EC:Xagawa22}, \cite[security.pdf: Section 5]{NISTPQC-R4:ClassicMcEliece22}.

    The $\prkey$ and $\mdsd$ assumptions are not needed here.
\end{proof}

\paragraph{Uniformity}

We analyze the public key and ciphertext uniformity of Classic McEliece and reduce to the two assumptions made above. As no further encoding is done here, there is no discussion of success rate as for other KEMs.
The notions of public key and ciphertext uniformity are discussed in \cref{def:pk-uniformity,def:ctxt-uniformity} respectively.

\begin{lemma}[Public key uniformity of Classic McEliece] \label{lem:classic-mceliece-pk-unif}
    Let $\textsf{CM}$ be the obfuscated KEM defined by the Classic McEliece specification \cite{NISTPQC-R4:ClassicMcEliece22} and using obfuscated key length $\textsf{ol} = mt \cdot k$ as well as obfuscated ciphertext length $\textsf{cl} = n-k$.
    
    For any adversary $\adv$ against the public key uniformity of $\textsf{CM}$, there exists an adversary $\bdv$ against the $\prkey$ assumption, such that
    \[ \advantage{\pkunif}{\textsf{CM}}[(\adv)] \leq \advantage{\prkey}{}[(\bdv)] \]
\end{lemma}
\begin{proof}
    The reduction $\bdv$ takes in a public key matrix $T_b$ and forwards its corresponding bitstring to $\adv$. Let $b$ be the output of $\adv$, then $\bdv$ returns the bit $1-b$.

    This perfectly simulates the $\pkunif$ game towards $\adv$ because the output of $\textsf{RandGen}()$ is statistically indistinguishable from $\textsf{RandGen}'()$ as shown in \cref{lem:classic-mceliece-randgen-prime}. The output of $\textsf{RandGen}'()$ corresponds to uniformly random bitstrings as required in $\pkunif$.

    The output bit needs to be flipped as $b=0$ corresponds to the real key generation in the game $\prkey$ but $b=1$ is used to denote the same in the $\pkunif$ game.
\end{proof}

\begin{lemma}[Ciphertext uniformity of Classic McEliece] \label{lem:classic-mceliece-ctxt-unif}
    Let $\textsf{CM}$ be the obfuscated KEM defined by the Classic McEliece specification \cite{NISTPQC-R4:ClassicMcEliece22} and using obfuscated key length $\textsf{ol} = mt \cdot k$ as well as obfuscated ciphertext length $\textsf{cl} = n-k$.
    
    For any adversary $\adv$ against the ciphertext uniformity of $\textsf{CM}$, there exist adversaries $\bdv, \cdv$ against the $\prkey$ and $\mdsd$ assumptions respectively, such that
    \[ \advantage{\ctxtunif}{\textsf{CM}}[(\adv)] \leq 2 \cdot \left( \advantage{\prkey}{}[(\bdv)] + \advantage{\mdsd}{}[(\cdv)] \right) \]
\end{lemma}
\begin{proof}
    We proceed via a series of two game hops, and consider the following three games:
    \begin{itemize}
        \item Let $G_0$ denote the $\ctxtunif$ game as in \cref{def:ctxt-uniformity}
        \item Let $G_1$ denote the same game as $G_0$ but with the change that $\pk$ is generated using $\textsf{RandGen}()$
        \item Let $G_2$ denote the same game as $G_1$ but with the change that $c_1 \gets \bin^\textsf{cl}$
    \end{itemize}

    As $G_0$ denotes the original $\ctxtunif$ game, we have:
    \[ 2 \left| \Pr[G_0(\adv) = 1] - \frac{1}{2} \right| = \advantage{\ctxtunif}{\textsf{CM}}[(\adv)] \]

    To bound the difference in advantage between $G_0$ and $G_1$, notice that we can define an adversary $\bdv$ against the $\prkey$ assumption which simulates the $\ctxtunif$ games towards $\adv$ using a new random bit for that simulation, and using the public key received from its $\prkey$ challenger. If $b=0$ in this $\prkey$ game, then $\adv$ is playing in $G_0$, if $b=1$, then $\adv$ is playing in $G_1$. Let $\bdv$ output its guess for the bit $b$ in the $\prkey$ game as $b=1$ if and only if $\adv$ wins the simulated game.

    Thus, any change in the probability of $\adv$ winning in $G_0$ vs $G_1$ directly corresponds to a change in the probability of $\bdv$ winning the $\prkey$ game. This means:
    \[ \left| \Pr[G_0(\adv) = 1] - \Pr[G_1(\adv) = 1] \right| \leq \advantage{\prkey}{}[(\bdv)] \]

    The analogous argument applies to the switch from $G_1$ to $G_2$: we can define an adversary $\cdv$ against the $\mdsd$ assumption which simulates the (modified) $\ctxtunif$ games towards $\adv$. If $b=0$ in the $\mdsd$ game, then $\adv$ is playing in $G_1$, if $b=1$, then $\adv$ is playing in $G_2$ (note that we translate in the obvious way between vectors of bits and bitstrings of the appropriate length). Thus:
    \[ \left| \Pr[G_1(\adv) = 1] - \Pr[G_2(\adv) = 1] \right| \leq \advantage{\mdsd}{}[(\cdv)] \]

    Finally, because both $c_0$ and $c_1$ are uniformly random bitstrings in game $G_2$,
    \[ \left| \Pr[G_2(\adv) = 1] - \frac{1}{2} \right| = 0 \]

    Combining the above inequalities using the triangle inequality yields the required bound.
\end{proof}
