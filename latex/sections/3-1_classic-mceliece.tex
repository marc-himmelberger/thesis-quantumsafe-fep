\section{Classic McEliece as an Obfuscated KEM} \label{sec:obfuscating-classic-mceliece}

Throughout this \cref{sec:obfuscating-classic-mceliece}, we use the positive integer variables $m,n,t,k$ where $n < 2^m$ and $k=mt-n$ as defined in \cite{NISTPQC-R4:ClassicMcEliece22}. We work with matrices and vectors over $\mathbb F_2$ when examining the KEM and use bitstrings in the defintion and analysis of OKEMs. As the dimensions of the matrices and vectors are fixed for a given parameter set, we associate the matrices and vectors with their canonical encoding as bitstrings of appropriate length.

\paragraph{Original generation and distribution}
Classic McEliece \cite{NISTPQC-R4:ClassicMcEliece22} is a perfectly correct, code-based KEM in Round 4 of the NIST Post-Quantum Cryptography Standardization \cite{nist-standardization}. Although NIST announced in early 2025 that Classic McEliece would not be standardized as part of this process \cite{nist-ir-8545}. Like \cite{EC:Xagawa22}, we only consider the so-called "non-f" versions of the specified parameter sets which are interoperable with the f versions, and have simpler but slower key generation.
 
In Classic McEliece, public keys are the $k$ rightmost columns of a $mt \times n$ generator matrix in systematic form ($k = n - mt$). Ciphertexts are the $mt$ data bits corresponding to $n$-bit codewords with Hamming weight exactly $t$.

The generation of public keys is a multistep process starting from a 256-bit seed which is expanded using SHAKE256 into a secondary seed and raw values. If the raw values do not form a suitable systematic code, another attempt is started from the secondary seed, and so on. This key generation process, among other algorithms from the Round 4 specification of Classic McEliece are reproduced in \cref{fig:classic-mceliece-spec}.

For encapsulation, rejection-sampling is used to generate a uniformly random vector of Hamming weight exactly $t$ which is then mapped from the space of $n$-bit vectors to $mt=n-k$ bits using the generator matrix defined by the public key.

\begin{figure}
    \begin{pchstack}[boxed, center, space=0.5em]

\begin{pcvstack}[space=0.05em]
\procedure[linenumbering]{$\kgen()$ \titlecomment{\text{see \cite[Sec. 5.3]{NISTPQC-R4:ClassicMcEliece22}}}}{
    \label{ln:mceliece-keygen}
    \delta \getsr \bin^\ell \\
    X, \delta' \gets G(\delta) \pccomment{$\delta' \in \bin^\ell$} \\
    s, \alpha_0, \dots, \alpha_{q-1}, g \gets X \pcskipln \\
    \t  \text{where $s$ is an $n$-bit string,}  \pcskipln \\
    \t  \text{$\alpha_0, \dots, \alpha_{q-1}$ are distrinct elements of $\mathbb F_q$,}  \pcskipln \\
    \t  \text{$g \in \mathbb F_q[x]$ is monic, irreducible,} \pcskipln \\
    \t  \text{and $\deg(g)=t$} \pcskipln \\
    \t  \text{if any of the above fail, restart} \pcskipln \\
    \t\t    \text{from \cref{ln:mceliece-keygen} with } \delta=\delta' \\
    \Gamma \gets (g, \alpha_0, \dots, \alpha_{n-1}) \\
    T \gets \mathsf{MatGen}(\Gamma) \\
    \pcreturn \pk=T, \sk=(\delta, g, \alpha_0, \dots, \alpha_{q-1}, s)
}
\procedure[lnstart=6,linenumbering]{$\mathsf{MatGen}(\Gamma=(g, \alpha_0, \dots, \alpha_{n-1}))$ \kern-1em \titlecomment{\text{see \cite[Sec. 4.2]{NISTPQC-R4:ClassicMcEliece22}}}}{
    \tilde H \gets \left\{\alpha_j^i/g(\alpha_j)\right\}_{i,j} \pccomment{$\tilde H \in \mathbb F_q^{t \times n}$}\\
    \hat H \gets \text{expand each polynomial entry of } \tilde H \pcskipln \\
    \text{ into $m$ rows of bits denoting its coefficients} \\
    (I_{mt} \mid T) \gets \text{reduce } \hat H \text{ into systematic form} \\
    \text{if this fails } T \gets \bot \\ 
    \pcreturn T
}
\end{pcvstack}

\begin{pcvstack}[space=0.05em]
\procedure[lnstart=11,linenumbering]{$\encaps(\pk=T)$ \titlecomment{\text{see \cite[Sec. 4.3]{NISTPQC-R4:ClassicMcEliece22}}}}{
    e \getsr \mathsf{FixedWeight}() \\
    H \gets (I_{mt} \mid T) \\
    C \gets He \in \mathbb F_2^{mt} \\
    K \gets H(1,e,C) \\
    \pcreturn (C, K)
}

\procedure[lnstart=16,linenumbering]{$\mathsf{FixedWeight}()$ \titlecomment{\text{see \cite[Sec. 5.4]{NISTPQC-R4:ClassicMcEliece22}}}}{
    a_0, \dots, a_{t-1} \gets [0, n-1] \\
    \text{if not all $a_i$ are distinct, restart} \\
    e \gets (e_0, \dots, e_{n-1}) \pcskipln \\
    \t  \text{ where, for all } i: e_{a_i} = 1 \\
    \pcreturn e
}
\end{pcvstack}

\end{pchstack}

    \caption[
        A selection of algorithms from the Classic McEliece Round 4 specification.
    ]{
        A relevant selection of algorithms from the Classic McEliece Round 4 specification \cite{NISTPQC-R4:ClassicMcEliece22}. $q=2^m$ refers to the size of a polynomial field of characteristic $2$. $G$ refers to a pseudorandom bit generator mapping a string of $\ell$ bits to a string of $\geq n + mq + mt + \ell$ bits. For the sake of brevity, the many more technical details, as well as the semi-systematic form, are omitted.}
    \label{fig:classic-mceliece-spec}
\end{figure}

We define the same assumptions about computational hardness as in \cite[Definition~K.1]{EC:Xagawa22}. These have appeared in more general forms in previous literature studying the Niederreiter and McEliece cryptosystems, e.g. \cite{AC:CouFinSen01,EC:SaiXagYam18}. \Cref{fig:classic-mceliece-assumptions} shows corresponding game-based definitions of $\prkey, \mdsd$.
\begin{itemize}
    \item \textbf{PR-Key assumption:} It is computationally hard to distinguish real public keys from the rightmost columns of the systematic forms of uniformly random generator matrices (conditioned on the systematic form existing).
    \item \textbf{modified Decisional Syndrome Decoding assumption:} It is computationally hard to distinguish ciphertexts generated using uniformly random generator matrices from uniformly random $(n-k)$-bit vectors.
\end{itemize}

For both games $\mathsf{goal} \in \{\prkey, \mdsd\}$, we define the advantage of an adversary $\adv$ against these games as
\[
    \advantage{\mathsf{goal}}{}[(\adv)]
    := 2 \cdot \left| \Pr\left[\mathsf{goal}(\adv) \Rightarrow 1\right] - \frac{1}{2} \right|
    \leq \epsilon.
\]

\begin{figure}
    \begin{pchstack}[boxed, center, space=0.5em]

\begin{pcvstack}[space=0.05em]
\procedure[linenumbering]{$\textbf{GAME } \textsf{PR-Key}(\adv)$}{
    b \gets \bin \\
    (T_0,\sk) \gets \kgen() \\
    T_1 \gets \textsf{RandGen}() \\
    b' \gets \adv(T_b) \\
    \pcreturn \llbracket b' = b \rrbracket
}

\procedure[lnstart=5,linenumbering]{$\textbf{GAME } \textsf{mDSD}(\adv)$}{
    b \gets \bin \\
    \hat T \gets \textsf{RandGen}() \\
    e \gets \textsf{FixedWeight}() \\
    u_0 \gets \left[ I_{n-k} \mid \hat T \right] \cdot e \\
    u_1 \gets U(\mathbb F_2^{n - k}) \\
    b' \gets \adv(\hat T, u_b) \\
    \pcreturn \llbracket b' = b \rrbracket
}
\end{pcvstack}

\begin{pcvstack}[space=0.05em]
\procedure[lnstart=12,linenumbering]{$\textsf{RandGen}()$}{
    \hat T \gets \bot \\
    \pcwhile \hat T = \bot \pcdo \\
    \t  \hat H \gets U(\mathbb F_2^{mt \times n}) \\
    \t  \text{reduce } \hat H \text{ to systematic form } \left[ I_{mt} \mid \hat T \right] \pccomment{TODO: check dimensions} \\
    \label{ln:randgen-reject}
    \t  \text{if this fails } \hat T \gets \bot \\ 
    \pcendwhile \\
    \pcreturn \hat T
}

\procedure[lnstart=19,linenumbering]{$\textsf{RandGen}'()$}{
    \hat T \gets U(\mathbb F_2^{mt \times k}) \\ 
    \pcreturn \hat T
}

\procedure[lnstart=30,linenumbering]{$\kgen()$ \titlecomment{\text{see \cite{NISTPQC-R4:ClassicMcEliece22}}}}{
    \pcreturn \bot \pccomment{TODO}
}

\procedure[lnstart=50,linenumbering]{$\textsf{FixedWeight}()$ \titlecomment{\text{see \cite{NISTPQC-R4:ClassicMcEliece22}}}}{
    \pcreturn \bot \pccomment{TODO}
}
\end{pcvstack}

\end{pchstack}

    \caption[
        Indistinguishability games for Classic McEliece.
    ]{
        Indistinguishability games for Classic McEliece with $\kgen$ and \textsf{FixedWeight} as defined in \cref{fig:classic-mceliece-spec}. These games follow the definitions from \cite[Definition~K.1]{EC:Xagawa22}. Note that $n-k = mt$.}
    \label{fig:classic-mceliece-assumptions}
\end{figure}

\paragraph{Classic McEliece has uniformly random public keys}
Although we do not require uniformly random encodings of public keys for this thesis, we believe to have found an equivalent but simplified form of the $\prkey$ assumption, that we found worthy of inclusion.

Given the $\prkey$ assumption, public keys are computationally hard to distinguish from outputs of $\textsf{RandGen}()$. We claim that $\textsf{RandGen}()$ can be simplified to a new algorithm $\textsf{RandGen}'()$, also shown in \cref{fig:classic-mceliece-assumptions}, with statistical indistinguishability between these two algorithms. This claim would imply computational indistinguishability between Classic McEliece's public keys and the outputs of $\textsf{RandGen}'()$.

\begin{lemma}[Simplifying \textsf{RandGen}] \label{lem:classic-mceliece-randgen-prime}
    Let $\textsf{RandGen}(), \textsf{RandGen}'()$ be defined as in \cref{fig:classic-mceliece-assumptions}.
    The distributions $D_0 = \{ \hat T \gets \textsf{RandGen}() \}$ and $D_1 = \{ \hat T \gets \textsf{RandGen}'() \}$ are statistically indistinguishable, i.e. have statistical distance $\Delta(D_0, D_1) = 0$.
\end{lemma}

\begin{proof}
    We first examine the rejection sampling behavior around \cref{ln:randgen-reject}: A resampling occurs if and only if a reduction of $\hat H$ to systematic form (also known as \emph{reduced row echelon form} or \emph{row canonical form}) fails. This reduction, in turn, fails if and only if the $mt$ leftmost columns of $\hat H$ do not form a full rank square matrix.
    The probability for a $mt \times mt$ matrix with i.i.d. uniformly random entries over $\mathbb F_2$ to be of full rank is exactly~\cite{DBLP:journals/corr/SalmondGGC14}
    \[ p_\mathsf{FR} := \prod_{i=1}^{mt} \left( 1-2^{-i} \right). \]
    
    For the parameter sets used in Classic McEliece ($768 \leq mt \leq 1664$), this is \[ p_\mathsf{FR} \approx 0.288788 . \]
    Differences between the parameter sets are negligible and a strict upper bound of $p_\mathsf{FR} \leq \prod_{i=1}^{768} \left( 1-2^{-i} \right) \leq \prod_{i=1}^{8} \left( 1-2^{-i} \right) \leq 0.29$ applies. A proof for a tight lower bound eludes us although the numerical convergence is rapid.
    
    This calculation further supports our statement that resampling occurs exactly for values of $\hat H$ whose $mt$ leftmost columns are not full rank, because the resulting probability $p_\mathsf{FR}$ is the same as was experimentally reported in \cite[security.pdf: Section 4.2]{NISTPQC-R4:ClassicMcEliece22}.
    
    After understanding which reductions fail, we now consider the possible influence that the reduction to systematic form might have on the distribution of bits in the simulated public key $\hat T$.

    Consider an intermediate value of $\hat H$ which can be reduced to systematic form (i.e. has full rank in its leftmost columns and will thus pass the check on \cref{ln:randgen-reject}). This value exists precisely once for all terminating executions of $\textsf{RandGen}()$ as the last $\hat H$ sampled before returning a result.

    In general, the reduced row echelon form can be obtained using Gauss–Jordan elimination. This algorithm carries out a sequence of operations on the matrix consisting of swapping rows, adding rows onto others and scaling rows.
    Over $\mathbb F_2$, the scaling of rows is not needed and swapping of rows can be accomplished via a sequence of 3 row additions. Any Gauss-Jordan reduction to systematic form can thus be expressed as merely a sequence of row additions.
    The systematic form of $\hat H$ over $\mathbb F_2$, just like reduced row echelon forms over any field, is also unique. Thus it suffices to consider an arbitrary sequence of row additions transforming $\hat H$ to systematic form.

    Note in particular, that during such a sequence of row additions, the row rank of $\hat H$ must be preserved, as the systematic form has full row rank. It is thus impossible for the sequence to ever add any row onto itself, i.e. all row additions must be between distinct rows. Any addition of a row onto itself would lower the rank by 1 and, trivially, no row addition can increase the rank of a matrix.
    
    Due to the number of rows, a reduction is possible in at most $(mt)^2$ row additions: The diagonal can be filled with ones using at most $mt$ additions. After another $(mt)^2-mt$ additions at the latest, all other entries in the leftmost $mt$ columns are zeroes. To determine one such order of row additions which bring $\hat H$ into systematic form, it suffices to consider only the leftmost $mt$ columns.
    
    Let $M_0$ denote the rightmost columns of $\hat H$, which form a $mt \times k$ matrix of (by definition) uniformly random i.i.d. entries from $\mathbb F_2$, in particular, independent from the leftmost columns of $\hat H$.
    
    As the addition of any one row of uniformly random bits onto any other row of independent uniformly random bits produces a row of uniformly random i.i.d. bits, the resulting matrix after one row addition, say $M_1$, is statistically indistinguishable from $M_0$.
    
    Repeating this argument for a maximum of $(mt)^2$ times, we can conclude that $\hat T$ is statistically indistinguishable from a uniformly random $mt \times k$ matrix over $\mathbb F_2$ as in $\textsf{RandGen}'()$.
\end{proof}

\paragraph{Classic McEliece is already obfuscated}
We now return to our examination of ciphertext uniformity in Classic McEliece.

Ciphertext uniformity follows directly from the $\prkey$ and $\mdsd$ assumptions, and we show this reduction, and thus computational indistinguishability from uniformly random bit vectors below. Additionally, implementers should note that padding public keys and ciphertexts to the byte boundary may require \emph{randomized} padding. This is only needed if the number of bits is not evenly divisible by 8, and occurs only for the parameter set ``mceliece6960119''.  All other parameter sets have $mt = 0 \mod 8$. Padding of these values with zero bits (as suggested in the Classic McEliece specification) would trivially break the desired ciphertext uniformity.

As we require no KEM encoding to achieve ciphertext uniformity, there is no need to discuss first-encaps success probability.

\begin{lemma}[Ciphertext uniformity of Classic McEliece]
\label{lem:classic-mceliece-ctxt-unif}
    Let $\mathsf{CM}$ be the obfuscated KEM defined by the Classic McEliece specification \cite{NISTPQC-R4:ClassicMcEliece22} with obfuscated ciphertext length $\obfctxtlen = n-k$.

    For any adversary $\adv$ against the ciphertext uniformity of $\mathsf{CM}$, there exists an adversary $\bdv$ against the $\sprcca$ security of $\mathsf{CM}$ with respect to the simulator $\simulunif$, and adversaries $\cdv_1, \cdv_2$ against the $\prkey$ and $\mdsd$ assumptions respectively, such that
    \[
        \advantage{\ctxtunif}{\mathsf{CM}}[(\adv)]
        \leq \advantage{\sprcca}{\KEM,\simulunif}[(\bdv)]
        \leq \advantage{\prkey}{}[(\cdv_1)] + \advantage{\mdsd}{}[(\cdv_2)].
    \]
\end{lemma}
\begin{proof}
    This follows directly from \cref{lem:ctxt-unif-for-bijections} because we can define $\encodectxt: \mathcal{C} \to \bin^\obfctxtlen$ to simply map vectors to their canonical representation as bitstrings. This encoding function is deterministic, bijective and never fails, i.e. $\epsilon^\firstencapssuccess_{\mathsf{CM},\encodectxt} = 1$.

    A reduction from $\sprcca$ with $\simulunif$ to the $\prkey$ and $\mdsd$ assumptions is given in \cite[Theorem~K.1]{EC:Xagawa22}.
\end{proof}

\paragraph{Security of Classic McEliece}

We have seen that Classic McEliece, as specified in \cite{NISTPQC-R4:ClassicMcEliece22}, is an obfuscated KEM without requiring any special encoding of ciphertexts. Additionally, Classic McEliece satisfies the required KEM security notions:

\begin{theorem}[KEM Security of Classic McEliece]
    Classic McEliece, as a KEM, achieves $\indcca$ and $\sprcca$ security.
    $\sprcca$ security holds with respect to the simulator $\simulunif$.
\end{theorem}
\begin{proof}
    The $\indcca$ and $\sprcca$ security notions for KEMs defined in \cref{def:kem-security} match the established notions for KEMs of the same names. These have been shown for Classic McEliece in previous work \cite[Section~K]{EC:Xagawa22}, and \cite[security.pdf: Section 5]{NISTPQC-R4:ClassicMcEliece22}.
\end{proof}

However, because we need 